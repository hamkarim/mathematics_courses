% Euclidean Handout Number One Fall 2013
\documentclass{tufte-handout}

%\geometry{showframe}% for debugging purposes -- displays the margins

%%%% Packages to make things pretty
\usepackage{amsmath,amsthm}
\usepackage{booktabs}
\usepackage{graphicx}
\setkeys{Gin}{width=\linewidth,totalheight=\textheight,keepaspectratio}
\graphicspath{{graphics/}}
\usepackage{units}
\usepackage{fancyvrb}
\fvset{fontsize=\normalsize}
\usepackage{multicol}
\usepackage{pdfpages}

%%%% Theorem Evironments
\theoremstyle{definition}

\newtheorem{problem}{Problem}
\newtheorem{conjecture}[problem]{Conjecture}
\newtheorem*{definition}{Definition}
\newtheorem*{theorem}{Theorem}
\newtheorem{question}[problem]{Question}
\newtheorem{challenge}[problem]{Challenge}

%%%%%


\title{Euclidean Geometry}
\author[]{Class Conjectures}
\date{Fall 2013}

\begin{document}

\maketitle

\renewcommand{\theproblem}{\Alph{problem}}

\begin{conjecture}[O'Connell]
Let $A, B, D$ and $K$ be four points.
\marginnote{28 August: Arose from work on Conjecture 1.2}
If angle $AKB$ and angle $AKD$ are right angles, then the points $B, D$ and $K$ are collinear.
\end{conjecture}

\begin{conjecture}[Dolash]
Let $ABC$ be an isosceles triangle with $AB$ congruent $BC$.
\marginnote{28 August: Arose from work on Conjecture 1.1. Settled 30 August.}
The segment joining $B$ to the midpoint of the opposite side is also the angle bisector at $B$.
\end{conjecture}

\end{document}

%sagemathcloud={"zoom_width":100}