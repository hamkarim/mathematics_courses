\documentclass[12pt,letterpaper]{article}

\usepackage[utf8]{inputenc}
\usepackage[T1]{fontenc}
\usepackage{amsmath}
\usepackage{amsfonts}
\usepackage{amssymb}
\usepackage{amsthm}
\usepackage[left=2cm,right=2cm,top=2cm,bottom=2cm,headheight=22pt]{geometry}
\usepackage{fancyhdr}
\usepackage{setspace}
\usepackage{lastpage}
\usepackage{graphicx}
\usepackage{caption}
\usepackage{subcaption}

\theoremstyle{definition}
\newtheorem{question}{Question}
\newtheorem{example}{Example}
\newtheorem{exercise}[question]{Exercise}
\newtheorem*{challenge}{Challenge}

\begin{document}

%Paramètres de mise en forme des paragraphes selon les normes françaises
\setlength{\parskip}{1ex plus 0.5ex minus 0.2ex}
\setlength{\parindent}{0pt}

%Paramètres relatifs aux en-têtes et pieds de page.
\pagestyle{fancy}
\lhead{Theron J Hitchman}
\chead{\Large Reading and Guided Practice \#2}
\rhead{Fall 2013}
\lfoot{\emph{Math and Decision Making}}
\cfoot{} 
\rfoot{\emph{\thepage\ of \pageref{LastPage}}}

\section*{Introduction}
In this assignment you will learn about how to see some physical properties of the picture hanging puzzle can be translated into properties of the symbols you learned about in previous work.
This should be helpful in trying to find a solution.

\section*{Goals}
At the end of this assignment, a student should be able to:
\begin{itemize}
\item state the effect of removing a nail from an attempted solution on the symbol associated to that attempt.
\item describe two situtations when attempted solutions are really the same physical solution, even though the symbols are different.
\end{itemize}
It is possible that a student might also:
\begin{itemize}
\item Solve the 2 nail picture hanging puzzle.
\end{itemize}

\section*{Reading and Questions for 28 August}




\begin{thebibliography}{9}

 \end{thebibliography}

\end{document}