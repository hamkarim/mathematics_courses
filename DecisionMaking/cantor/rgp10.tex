\documentclass[12pt,letterpaper]{article}

\usepackage[utf8]{inputenc}
\usepackage[T1]{fontenc}
\usepackage{amsmath}
\usepackage{amsfonts}
\usepackage{amssymb}
\usepackage{amsthm}
\usepackage[left=2cm,right=2cm,top=2cm,bottom=2cm,headheight=22pt]{geometry}
\usepackage{fancyhdr}
\usepackage{setspace}
\usepackage{lastpage}
\usepackage{graphicx}
\usepackage{caption}
\usepackage{subcaption}
\usepackage{paralist}
\usepackage{url}

\theoremstyle{definition}
\newtheorem{question}{Question}
\newtheorem{example}{Example}
\newtheorem{exercise}[question]{Exercise}
\newtheorem*{challenge}{Challenge}
\newtheorem*{theorem}{Theorem}
\newtheorem*{definition}{Definition}
\newtheorem*{observation}{Observation}

\begin{document}

%Paramètres de mise en forme des paragraphes selon les normes françaises
\setlength{\parskip}{1ex plus 0.5ex minus 0.2ex}
\setlength{\parindent}{0pt}

%Paramètres relatifs aux en-têtes et pieds de page.
\pagestyle{fancy}
\lhead{Theron J Hitchman}
\chead{\Large Reading and Guided Practice \#10}
\rhead{Spring 2014}
\lfoot{\emph{Math and Decision Making}}
\cfoot{}
\rfoot{\emph{\thepage\ of \pageref{LastPage}}}

\section*{Introduction}
In this reading, we introduce a model for the \emph{real numbers}. Also, we create a notational system for real numbers and explore some of its basic properties.

\section*{Goals}
At the end of this assignment, a student should be able to:
\begin{compactitem}
\item Give a definition of the set of real numbers.
\item Given a real number, find the decimal notation for it.
\item Given a decimal notation, find the associated real number.
\item Describe the trouble with decimal notation.
\end{compactitem}

\section*{Reading and Questions for Cantor's Paradise Meeting 11}

So far we have encountered the natural numbers, $\mathbb{N}$, the integers, $\mathbb{Z}$, and the rational numbers, $\mathbb{Q}$. 
We have also encountered a thing which we want to be a number, $\sqrt{2}$, but is not a rational number.
Somehow, we need a bigger set of numbers.

A hint lies in the way we usually arrange the numbers.
We tend to imagine the natural numbers as being a long string of telephone poles along a country highway.
Somehow, we have to imagine this set of things goes on ``to infinity.''
The integers are similar, except they stretch out in both directions along the line.

The rational numbers are situated in the gaps.
The number $1/2$ sits halfway between $0$ and $1$.
The number $-4/3$ sits a third of the way from $-1$ to $-2$ (moving left this time).

\begin{exercise}
Draw a number line and place the numbers $0$, $1$, $1/2$, $-1$, $-2$, and $-4/3$ on it.
Be sure to place them accurately.
\end{exercise}

But where shall we place a number like $\sqrt{2}$?

[Hang on to your hats, folks.
This is what you learned in school, but it might not have been said quite this way.]

\subsection*{More Numbers}

So, now we fix a special line, $\ell$ and two points $O$ and $I$ on that line.

\begin{definition}
A \emph{real number} is a point on the line $\ell$.
The set of all real numbers is denoted $\mathbb{R}$.
\end{definition}

That is a crazy definition, is it not?
The geometric approach will make what we want to do much simpler.
(On the other hand, it makes things like adding and multiplying much more challenging. 
That is not a concern for us, because we will not have much use for the arithmetic of real numbers.)

We are going to need a way to understand how to pick out particular real numbers and describe them.

\subsection*{Geometric Positional Notation: The Decimal System}

Our system of notation for integers relies on using the ten symbols of the Hindu-Arabic numerals: $\{0, 1, 2, 3, 4, 5, 6, 7, 8, 9\}$.
Now we shall develop a way use some notation to pick out a particular number.
Since we have ten standard symbols, we shall use a geometric system based on dividing things into tenths.

To keep things simple we will work only with the set $\mathcal{R}$ of all real numbers which lie between $O$ and $I$.

For each real number, that is for each point in the interval $\mathcal{R}$, we will assign a symbol of the form 
\[
0.a_1a_2a_3\ldots a_n\ldots,
\]
where each of the $a_i$'s is one of allowed digits $0, 1, 2, 3, 4, 5, 6, 7, 8, 9$.
The important thing here is that the symbol has as many $a_i$'s as there are natural numbers!

Suppose one is given a point $P$ lying in $\mathcal{R}$.
We now describe the way to find the symbol for $P$.

First, divide $\mathcal{R}$ into ten subintervals of equal length.
These subintervals are labelled with the ten digits of our numeral system, in order, from left to right.

\begin{exercise}
Draw a diagram of the interval $R$ and its subdivision into ten equal length subintervals, with appropriate labels.
\end{exercise}

The point $P$ lies inside one of these subintervals. 
The label on that subinterval is our $a_1$.

Now, we take that subinterval and further divide it into ten more equal length subintervals. 
Label these second level subintervals with the digits, in order, from left to right.
We choose the number $a_2$ by whichever subinterval of this $P$ lies in.

And this process continues over an over again, at each stage a new digit is chosen by which of the ten subintervals the point $P$ lies in.

\begin{example}
If the point $P$ lies in the third subinterval of $\mathcal{R}$ and then in the  the fifth subinterval of that, then its decimal notation starts with $0.24$.
\end{example}

\begin{exercise}
Draw a diagram that explains the last example.
\end{exercise}

\begin{exercise}
Suppose that the point $P$ lies in the first subinterval of $\mathcal{R}$ and then in the tenth and last subinterval of that.
Draw a diagram that explains why the decimal notation for $P$ begins $0.09$.
\end{exercise}

\begin{exercise}
Suppose that the point $P$ lies in the third subinterval of $\mathbb{R}$, then the third subinterval of that, and then the third subinterval of that.
What is the beginnig of the decimal notation for $P$?
\end{exercise}

\begin{challenge}
Dividing up $\mathcal{R}$ into tenths, and then dividing those intervals into tenths leads to a situation where $\mathcal{R}$ is divided up into one hundred equal length subintervals.
If we label them by the numbers $a_1a_2$ from our system, what happens?
Make a diagram and describe what you see.
\end{challenge}

\begin{exercise}
Find the complete decimal notation for the points $O$ and $I$.
\end{exercise}


\subsection*{Finding a Given Decimal in $\mathcal{R}$}

Of course, it is possible to go the other way, too.
If you are given a real number already described in decimal notation, it is not too difficult to find where on the line it lives.
Just work backward!
The notation tells us how to successively locate which subintervals our point lies in.
As we use more digits, we get more precision, because the subinterval shrink at each stage.

\begin{example}
A point with decimal notation $0.125\ldots$ lies in the sixth subinterval of the third subinterval of the second subinterval of $\mathcal{R}$.
\end{example}

\begin{exercise}
Make a diagram that explains the last example.
\end{exercise}

\begin{exercise}
Suppose that we divide $\mathcal{R}$ into one thousand subintervals of equal length.
To which of these does a point with notation $0.125\ldots$ belong?
\end{exercise}



\subsection*{A Little Spot of Trouble}

All of this works just great\dots expect for a little bit of trouble.
What do we do about the points which lie in two intervals?
For example, consider the midpoint $M$ of $\mathcal{R}$, which lies halfway between $O$ and $I$.

It is possible to see $M$ as lying in \emph{both} the fifth and sixth subintervals of $\mathcal{R}$, because it is the boundary between the two.
Depending on the choice, we get different decimal notations for $M$.
On one hand, we could start with $0.4\ldots$ and on the other we can start with $0.5\ldots$.

Suppose we choose the fifth subinterval.
Then $M$ is the farthest right hand point for the whole rest of the process.
Every other choice we make is forced to be the tenth subinterval, so we get digits of $9$.
We see that $M$ has decimal notation $0.49999999\ldots$ where the $9$'s repeat the rest of the way.

Instead, suppose we choose the sixth subinterval of $\mathcal{R}$.
Then $M$ is the farthest left hand point for the whole rest of the process.
Every other choice we make is forced to be in the first subinterval, so we get digits of $0$.
We see that $M$ has decimal notation $0.5000000\ldots$ where the $0$'s repeat the rest of the way.


\begin{observation}
The decimal notation of a point $P$ in $\mathcal{R}$ can only be ambiguous if $P$ lies on the boundary between two subintervals at some point in the process.
In such a case, $P$ will have \textbf{two} notations, one will end with an infinite string of $9$'s and the other will end with an infinte string of $0$'s.
\end{observation}


\begin{exercise}
Consider the number $Q$ which is the midpoint of the segment $OM$.
This $Q$ lies halfway between $O$ and $M$, and has two decimal notations.
Find these two notations.
\end{exercise}



%\begin{thebibliography}{9}
%\end{thebibliography}

\end{document}
%sagemathcloud={"zoom_width":100}