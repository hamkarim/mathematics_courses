\documentclass[12pt,letterpaper]{article}

\usepackage[utf8]{inputenc}
\usepackage[T1]{fontenc}
\usepackage{amsmath}
\usepackage{amsfonts}
\usepackage{amssymb}
\usepackage{amsthm}
\usepackage[left=2cm,right=2cm,top=2cm,bottom=2cm,headheight=22pt]{geometry}
\usepackage{fancyhdr}
\usepackage{setspace}
\usepackage{lastpage}
\usepackage{graphicx}
\usepackage{caption}
\usepackage{subcaption}
\usepackage{paralist}

\theoremstyle{definition}
\newtheorem{question}{Question}

\begin{document}

%Paramètres de mise en forme des paragraphes selon les normes françaises
\setlength{\parskip}{1ex plus 0.5ex minus 0.2ex}
\setlength{\parindent}{0pt}

%Paramètres relatifs aux en-têtes et pieds de page.
\pagestyle{fancy}
\lhead{Theron J Hitchman}
\chead{\Large Second Exam}
\rhead{Fall 2013}
\lfoot{\emph{Math and Decision Making}}
\cfoot{}
\rfoot{\emph{\thepage\ of \pageref{LastPage}}}

\section*{Instructions}
Please write your responses, with any supporting diagrams, on the separate pieces of white paper provided.  You have one hour to complete this exam. I expect you to explain yourself clearly and in complete sentences whenever possible.

\section*{The Examination Questions}

\begin{question}
Let $\mathcal{M}$ be the set of all arrangements of the letters of the word MISSISSIPPI.
Figure out what typical elements of $\mathcal{M}$ look like, and what it means for something to not be an element of $\mathcal{M}$.
Describe this in a sentence or two.

Then give a list of five examples of elements of $\mathcal{M}$ and two examples of things that someone who is not being careful enough would say are elements of $\mathcal{M}$ but really are not.
\end{question}

%\vspace{.5in}

%\begin{question}
%Let $B_{10}$ be the set of all mathematical words of length $10$, where the only ``letters'' used are $0$ and $1$.
%Let $\mathcal{P}(\lfloor 10 \rfloor)$ be the set of all subsets of the set $\lfloor 10 \rfloor = \{ 1, 2, 3, 4, 5, 6, 7, 8, 9, 10 \}$.
%Describe a matching between the elements of $B_{10}$ and the elements of $\mathcal{P}(\lfloor 10 \rfloor)$.
%\end{question}

\vspace{.5in}

\begin{question}
Recall that $\mathcal{E}$ is the set of all even natural numbers.
Let $\mathcal{F}$ be the set of all natural numbers which are multiples of $4$.
These sets are both countably infinite, and hence ``have the same size.''
Describe a matching between the elements of $\mathcal{E}$ and the elements of $\mathcal{F}$.
\end{question}

\vspace{.5in}

\begin{question}
Explain why the set of positive rational numbers is countably infinite.
Be sure to give a complete justification.

What about this result is surprising?
\end{question}

\vspace{.5in}

\begin{question}
Explain why the set $\mathcal{R}$ of real numbers between $0$ and $1$ is not countably infinite.

What about this result is surprising?
\end{question}

\clearpage











\end{document}
%sagemathcloud={"zoom_width":100}