\documentclass[12pt,letterpaper]{article}

\usepackage[utf8]{inputenc}
\usepackage[T1]{fontenc}
\usepackage{amsmath}
\usepackage{amsfonts}
\usepackage{amssymb}
\usepackage{amsthm}
\usepackage[left=2cm,right=2cm,top=2cm,bottom=2cm,headheight=22pt]{geometry}
\usepackage{fancyhdr}
\usepackage{setspace}
\usepackage{lastpage}
\usepackage{graphicx}
\usepackage{caption}
\usepackage{subcaption}
\usepackage{paralist}
\usepackage{url}

\theoremstyle{definition}
\newtheorem{question}{Question}
\newtheorem{example}{Example}
\newtheorem{exercise}[question]{Exercise}
\newtheorem*{challenge}{Challenge}
\newtheorem*{theorem}{Theorem}
\newtheorem*{definition}{Definition}

\begin{document}

%Paramètres de mise en forme des paragraphes selon les normes françaises
\setlength{\parskip}{1ex plus 0.5ex minus 0.2ex}
\setlength{\parindent}{0pt}

%Paramètres relatifs aux en-têtes et pieds de page.
\pagestyle{fancy}
\lhead{Theron J Hitchman}
\chead{\Large Reading and Guided Practice \#2}
\rhead{Spring 2014}
\lfoot{\emph{Math and Decision Making}}
\cfoot{}
\rfoot{\emph{\thepage\ of \pageref{LastPage}}}

\section*{Introduction}
In this assignment, you will learn about \emph{subsets}. This will give us flexibility to build many new examples and describe them carefully. Also, you will learn about a special  construction, the \emph{power set} of a set.

\section*{Goals}
At the end of this assignment, a student should be able to:
\begin{compactitem}
\item Describe clearly the meaning of the word subset.
\item Decide if one set is, or is not, a subset of another.
\item Model some counting problems using the language of subsets.
\end{compactitem}
A student might also be able to:
\begin{compactitem}
\item Show that some counting problems involving subsets are very similar.
\end{compactitem}

\section*{Reading and Questions for Cantor's Paradise Meeting 3}

\subsection*{The Notion of a Subset}
Suppose you have a crowd of $50$ people, and from that crowd you must choose two people to be representatives for a mission to Mars. 
How many ways can you make such a choice? 
The number is rather large, but for now we do not care what it is. 
More important to us is the structure of what is going on.

Initially, you have a set of $50$ people like this:
\[
C = \{ \text{Jane}, \text{Joe}, \text{Jessica}, \text{Jeremey}, \ldots, \text{Jennifer} \}.
\]
(Recall that the ellipsis, that is the three periods in a row, indicates that some elements are missing from the description.)
A choice of a pair of people to go to Mars is the a set like $\{ \text{Jane}, \text{Joe} \}$ or $\{\text{Jeremey}, \text{Jennifer} \}$. 
Each of those things is a \emph{subset} of the original set $C$.
What does this mean?

\begin{definition} Let $S$ be a set. Another set $R$ is called a \emph{subset} of $S$ when for each element of $R$, that thing is also an element of $S$.
\end{definition}

Why is $\{ \text{Jane}, \text{Joe}\}$ a subset of $C$?
Just check!
The elements of $\{ \text{Jane}, \text{Joe} \}$ are $\text{Jane}$ and $\text{Joe}$. Each of these is an element of $C$, also. Therefore, by the definition above, $\{ \text{Jane}, \text{Joe}\}$ is a subset of $C$.

\begin{exercise}
Give two other examples of subsets of $C$ and describe how you know they are subsets.
\end{exercise}

\begin{exercise}
Find two more subsets of $C$, this time one of your subsets should have less than two elements, and one should have more than two elements.
\end{exercise}

\subsection*{Some Exercises}

\begin{exercise}
Consider the set $X = \{A, 3, \{1\} \}$.
List as many subsets of $X$ as you can.
(There are eight of these. How many can you find?)
\end{exercise}

That last exercise is tricky.
We will come back to count all of the subsets in just a bit.

\begin{exercise}
Let $T = \{1, 2\}$ and $J = \{1, 3\}$ be two sets.
Use the definition to write some sentences that explain why $T$ is not a subset of $J$ and $J$ is not a subset of $T$.
\end{exercise}

Constructions of subsets allows us great flexibility but it is important to be careful. For the next few exercises, we will work with the set $G$,
\[
G = \{ \{1, 2\}, \{3, 7\}, \{1, 3\} \}.
\]

\begin{exercise}
Use the definition of subset to explain why $H = \{ \{1,2\}, \{1, 3\} \}$ is a subset of $G$.
\end{exercise}

\begin{exercise}
Use the definition of subset to explain why $K = \{ 1, 2, 3\}$ and $B = \{ 7\}$ are not subsets of $G$.
\end{exercise}

\subsection*{Oddities of Subsets}

Mathematicians have found it useful to make some conventions that seem odd to the newcomer.

\subsubsection*{The Empty Set}
First, there is a special set called the \emph{empty set}. 
This is the set that \emph{has no elements}! 
The standard way to write this is with this symbol $\emptyset$, but really that is just shorthand for $\{ \quad \}$. 

The goofiest thing about the empty set is that it is a subset of \underline{every} other set.
Why?
Think about the definition: for each element of $\emptyset$ we need something to be true.
But there are no elements of $\emptyset$.
So we get that whatever we need is true because there is nothing to check.

\subsubsection*{The Whole Subset}
Suppose you have some set $S$.
If you work through the definition of the word subset, you can see that $S$ is a subset of itself.
This is a bit weird because our intuition usually says that a subset should be something smaller than the original.
But this is the way the definition works. Any set is a subset of itself.

\begin{exercise}
Go back to the exercise about finding all of the subsets of $X = \{A, 3, \{1\} \}$.
Can you find all eight of them now?
\end{exercise}

\subsubsection*{The Power Set of a Set}
Collecting subsets together allows us to make an important construction.
\begin{definition}
Let $S$ be a set. The \emph{power set of} $S$ is the set whose elements are all of the subsets of $S$.
\end{definition}

The power set is usually denoted by $\mathcal{P}$. So, in set builder notation, this definition looks like this:
\[
\mathcal{P}(S) = \{ X \mid \text{$X$ is a subset of $S$} \}
\]

Here we have extended the set-builder notation scheme. Inside the curly braces, you find a description with some name in it (here it is $X$), then a vertical bar, and then a description of the rule one uses to decide if the name belongs to the set or not.

\begin{exercise}
Let $F = \{1, 2, 3, 4\}$ be a set.
Describe the power set $\mathcal{P}(F)$ of $F$ by writing down all of its elements.
Note that there are sixteen of them.
\end{exercise}

\subsection*{Challenges}

Here are some tasks that will help you practice common things we will want to do with subsets.

\begin{challenge}
In a city of $4000$ people, we are to choose some to get shiny gold medals with pictures of Prof.~Hitchman's distinguished face on one side, and the number 403 on the other.
We want to pick three special people to get medals.
\begin{enumerate}
\item[(a)] Make a definition of a set $T$ so that its elements represent possible choices of three people to \emph{get} medals.
You should use the word ``subset'' somehow.
\item[(b)] Make a definition of a set $J$ so that its elements represent possible choices of people to \emph{not} get medals.
You should use  the word ``subset'' somehow.
\item[(c)] [Extra challenge] How can we see that $T$ and $J$ have the same number of elements without counting?
\end{enumerate}
\end{challenge}

\begin{challenge}
Let $\mathbb{N}$ be the set of natural numbers, and let $\mathcal{E}$ be the set of all even natural numbers.
\[
\mathbb{N} = \{1, 2, 3, 4, \ldots \}, \qquad \mathcal{E} = \{2, 4, 6, 8, \ldots \}
\]
Use the definition of subset to check that $\mathcal{E}$ is a subset of $\mathbb{N}$, but $\mathbb{N}$ is not a subset of $\mathcal{E}$. 

Which of these sets is bigger? 
\end{challenge}


%\begin{thebibliography}{9}
%\end{thebibliography}

\end{document}
%sagemathcloud={"zoom_width":100}