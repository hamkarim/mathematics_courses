\documentclass[12pt,letterpaper]{article}

\usepackage[utf8]{inputenc}
\usepackage[T1]{fontenc}
\usepackage{amsmath}
\usepackage{amsfonts}
\usepackage{amssymb}
\usepackage{amsthm}
\usepackage[left=2cm,right=2cm,top=2cm,bottom=2cm,headheight=22pt]{geometry}
\usepackage{fancyhdr}
\usepackage{setspace}
\usepackage{lastpage}
\usepackage{graphicx}
\usepackage{caption}
\usepackage{subcaption}
\usepackage{paralist}
\usepackage{url}

\theoremstyle{definition}
\newtheorem{question}{Question}
\newtheorem{example}{Example}
\newtheorem{exercise}[question]{Exercise}
\newtheorem*{challenge}{Challenge}
\newtheorem*{theorem}{Theorem}
\newtheorem*{definition}{Definition}

\begin{document}

%Paramètres de mise en forme des paragraphes selon les normes françaises
\setlength{\parskip}{1ex plus 0.5ex minus 0.2ex}
\setlength{\parindent}{0pt}

%Paramètres relatifs aux en-têtes et pieds de page.
\pagestyle{fancy}
\lhead{Theron J Hitchman}
\chead{\Large Reading and Guided Practice \#1}
\rhead{Spring 2014}
\lfoot{\emph{Math and Decision Making}}
\cfoot{}
\rfoot{\emph{\thepage\ of \pageref{LastPage}}}

\section*{Introduction}
This assignment focuses on the basic language of set theory.

\section*{Goals}
At the end of this assignment, a student should be able to:
\begin{compactitem}
\item Demonstrate proper usage of the terms \emph{set}, \emph{element}, and \emph{member}.
\item Distinguish whether a particular object is, or is not, an element of a given set.
\item Use set builder notation to describe a set.
\end{compactitem}
A student might also be able to:
\begin{compactitem}
\item Solve a challenging problem about some large sets.
\end{compactitem}

\section*{Reading and Questions for Cantor's Paradise Meeting 2}

In this unit, we shall study the concept of infinity. 
In particular, we shall explore different concepts of ``number'' and think about how big different collections of numbers are. 
Numbers began as a way to count things and keep track of how many things you have, so we shall encounter some challenges about counting things.

To talk about all of this clearly and coherently, we will have to be very precise about our use of language.
The first task is to understand the concepts of \emph{set}, and \emph{element}.

\subsection*{Sets and Elements}

The words \emph{set} and \emph{element} are undefined terms in mathematics.
They are considered so foundational that we can't clearly say exactly what they are.
Instead, we can just give some intuitive idea of them.

A \emph{set} is a collection of some things. Those things in the collection are called the \emph{elements} of that set.
The only way to use these words is to say either
\begin{quote}
The object $x$ is an element of the set $S$.
\end{quote}
or
\begin{quote}
The object $x$ is not an element of the set $S$.
\end{quote}

\clearpage

Sometimes, the word \emph{member} is used in place of \emph{element}. 
In this context, \emph{member} and \emph{element} are synonyms.
If we make this switch, our phrases read
\begin{quote}
The object $x$ is a member of the set $S$.
\end{quote}
or
\begin{quote}
The object $x$ is not a member of the set $S$.
\end{quote}
\textbf{Membership is the only thing that matters with sets.} Either your object is a member of the set, or it is not.

\begin{example} We give two new examples of sets.
\begin{compactitem}
\item The collection of all students enrolled at UNI is a set.
Each enrolled student is a member of this set.
Prof.~Hitchman is not a member of this set.
The chair on which you are sitting is also not a member of this set.

\item The collection of all even integers is a set. We shall denote this set by the symbol $\mathcal{E}$.
The numbers $4$, $18$, and $416$ are all elements of $\mathcal{E}$, though not the only ones, by far. 
The number $227$ is not an element of $\mathcal{E}$.
\end{compactitem}
\end{example}

\begin{exercise}
Note that in the two examples above, one uses the terminology "member of" and the other uses the terminology "element of." Rewrite these sentences by hand, switching the terms around. 
(Yes. Seriously. Write it out. It will help.)
\end{exercise}

\subsection*{Notation for Sets}

There are two very common ways to describe a set.
The first way is just to use regular sentences.
We did this in the two examples above.
To make another example, let us first introduce the idea of an arrangement.


\begin{definition}
Suppose you have some things. 
An \emph{arrangement} of those things is a particular way to make a list out of exactly those things in some order, without repeats and without skipping any.
\end{definition}

For example, an arrangement of the athletic department mascots for UNI is 
\begin{quote}
TC~TK.
\end{quote}
A different arrangement of those same mascots is
\begin{quote}
TK~TC.
\end{quote}

\begin{example}
Let $T$ be the set of all arrangements of the letters A, B, and C.
\end{example}

\begin{exercise}
Write down three different elements of the set $T$.
\end{exercise}

\textbf{Do not read any further until you have attempted that exercise.}

\clearpage

\noindent\textbf{Solution:} 
What is an element of $T$? An arrangement of the letters of ABC is an element. Well, ABC is one arrangement.
Then BAC is another one. 
And BCA yet another.
Can you find a few more?

\begin{exercise}
Suppose that on a short trip you pack the following clothes in your bag:
\begin{compactitem}
\item three shirts: red, blue, and green
\item two pairs of pants: blue, and black-and-white checkerboard
\end{compactitem}
Let $J$ be the set of outfits you canmake by pairing a shirt with a pair of pants.

List all of the elements of $J$.
\end{exercise}

\begin{exercise}
Which is larger, $T$ or $J$?
Or are they the same size?
\end{exercise}

Another way to describe a set is by using ``set builder notation.''
Here you give the name of the set, write an equality symbol, and then list the elements of the set inside a pair of curly braces.

\begin{example}
$S = \{ 3, 7, 12\}$ is a small set with three elements.
\end{example}

One thing that is a bit tricky is that your list really should not repeat. If it does, then the repeats do not count as new elements. Remember, all that matters is "Is $x$ a member or is it not a member?" You cannot have an element that is a member more than once. Once is all you get.

\begin{example}
The sets
\[
\{1,1,2,3,5,8,13,21\}
\] 
and 
\[ 
\{1,2,3,5,8,13,21\}
\]
are exactly the same thing. The number $1$ is an element of both. The repeated $1$ in the list for the first set is just a nuisance.
\end{example}

Some lists become too long to be convenient, so people often use an ellipsis (\dots) to imply that some pattern repeats to describe members of a set.

\begin{example}
The set of natural numbers is
\[
\mathbb{N} = \{1,2,3,4,\ldots\}.
\]
\end{example}
\begin{example}
The set $T =\{1, 4, 7, 10, 13,\ldots, 3001\}$ has implied members.
\end{example}

\begin{exercise}
Give examples of two elements of $T$ that are not explicitly shown in the set builder notation.
\end{exercise}

\begin{exercise}
Give examples of two natural numbers which are \textbf{not} members of $T$.
\end{exercise}

\begin{challenge}
Let $T$ be the set of all arrangments of the letters ABCDEFGHIJ.
Let $J$ be the set of all arrangments of the letters KLMNOPQRST.
Write down some elements of $T$.
Write down some elements of $J$.

Are these sets the same size?
Or is it the case that one is larger than the other?
If one is larger? 
Which one?
\end{challenge}



%\begin{thebibliography}{9}
%\end{thebibliography}

\end{document}
%sagemathcloud={"zoom_width":100}