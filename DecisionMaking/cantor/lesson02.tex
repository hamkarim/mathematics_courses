\documentclass[12pt]{amsart}
\usepackage[margin=1in]{geometry}
\usepackage{paralist}

\theoremstyle{definition}
\newtheorem{question}{Question}

\begin{document}
\begin{center}
\textbf{\Huge
Lesson Plan: Meeting Two
}
\end{center}
\vspace{.5in}

Do Peer Instruction on the concepts of set and element.

\begin{question}
How many elements does the set $\{ ABC, BAC, D \}$ have?
\begin{compactitem}
\item two
\item three
\item four
\end{compactitem}
\end{question}

\begin{question}
Let $X$ be the set of all arrangements of the letters ABCD.
How many elements does $X$ have?
\begin{compactitem}
\item four
\item ten
\item twenty
\item twenty-four
\end{compactitem}
\end{question}

\begin{question}
Consider the set $Y = \{ 7, \{0, 1\}, 2\}$.
How many elements does $Y$ have?
\begin{compactitem}
\item two
\item three
\item four
\end{compactitem}
\end{question}

\begin{question}
Consider the set $Z = \{ 0, 3\times 5, 6, 15, 6\}$.
How many elements does $Z$ have?
\begin{compactitem}
\item two
\item three
\item four
\item five
\end{compactitem}
\end{question}

\begin{question}
How many elements does the set $W = \{ 2, 4, 6, \ldots, 14 \}$ have?
\begin{compactitem}
\item four
\item five
\item six
\item seven
\item more than seven
\end{compactitem}
\end{question}


\begin{question}
Monday we had 56 people in class. % Note: change this each semester!
We noticed that there were 1540 ways to form a two person team by choosing people from class.
Then we asked how many ways can one form a 54 person team by choosing people from class.
This number is
\begin{compactitem}
\item less than 1540.
\item equal to 1540.
\item greater than 1540.
\end{compactitem}
\end{question}


\end{document}














%sagemathcloud={"zoom_width":100}