\documentclass[12pt,letterpaper]{article}

\usepackage[utf8]{inputenc}
\usepackage[T1]{fontenc}
\usepackage{amsmath}
\usepackage{amsfonts}
\usepackage{amssymb}
\usepackage{amsthm}
\usepackage[left=2cm,right=2cm,top=2cm,bottom=2cm,headheight=22pt]{geometry}
\usepackage{fancyhdr}
\usepackage{setspace}
\usepackage{lastpage}
\usepackage{graphicx}
\usepackage{caption}
\usepackage{subcaption}
\usepackage{paralist}
\usepackage{url}

\theoremstyle{definition}
\newtheorem{question}{Question}
\newtheorem{example}{Example}
\newtheorem{exercise}[question]{Exercise}
\newtheorem*{challenge}{Challenge}
\newtheorem*{theorem}{Theorem}
\newtheorem*{definition}{Definition}
\newtheorem{observation}{Observation}

\begin{document}

%Paramètres de mise en forme des paragraphes selon les normes françaises
\setlength{\parskip}{1ex plus 0.5ex minus 0.2ex}
\setlength{\parindent}{0pt}

%Paramètres relatifs aux en-têtes et pieds de page.
\pagestyle{fancy}
\lhead{Theron J Hitchman}
\chead{\Large Reading and Guided Practice \#5}
\rhead{Spring 2014}
\lfoot{\emph{Math and Decision Making}}
\cfoot{}
\rfoot{\emph{\thepage\ of \pageref{LastPage}}}

\section*{Introduction}
In this reading we reassess the idea of counting and compare it to our idea of matching.

\section*{Goals}
At the end of this assignment, a student should be able to:
\begin{compactitem}
\item Explain how counting is making a matching with some initial segment of the natural numbers.
\item Explain how counting compares with making a list.
\item Explain the importance of the order given to the elements of the list.
\end{compactitem}
A student might also be able to:
\begin{compactitem}
\item Solve some challenging problems about counting by using lists appropriately.
\end{compactitem}

\section*{Reading and Questions for Cantor's Paradise Meeting 6}

So far, we have tried to learn to compare the sizes of sets \emph{without} actually counting them.
The idea of comparing sets by attempting to match them up is an old and powerful one, but of course it is not the only one.
Let us revisit the idea of counting with a fresh perspective.

\subsection*{A New Look at Counting}

Begin with the following simple exercise.
Pay attention to \emph{how} you solve it.
\begin{exercise}
Consider the set $C = \{ \pi, e, 7, 22, 948, 51, 403\}$.
How many elements does $C$ have?
\end{exercise}

\vspace{.5in}

What did you do when you completed that exercise?
I am sure you counted the elements.
Did you touch each element of $C$ with a pencil while reciting the names of numbers in the proper order?
Maybe you did not physically touch the elements, but instead just let your eye linger over each as you said the names of numbers in your head?
\begin{quote}
\emph{One, two, three, four, five, six, seven.}
\end{quote}
What you just did was to make a matching between the elements of $C$ and the set $\{1, 2, 3, 4, 5, 6, 7\}$.
You are not accustomed to thinking of it that way, but that is what happened.

This is how counting works!
When we teach children to count, there are really two things happening.
\begin{compactdesc}
\item[(Obvious)] We help children commit to memory our standard scheme for naming numbers in a particular order.
\item[(Not obvious)] We teach children the implicit process of making a matching between the collection of things to be counted and some properly chosen subset of the natural numbers.
\end{compactdesc}

The second part lies deeper in the process.
Even though it is usually done implicitly, children pick it up at some point.
I know a four year old who has the process picked up, she always makes matchings, now.
But she has not, yet, mastered the scheme for naming numbers. 
In a way this is reasonable, the naming scheme has some inherent order to it, but it really is arbitrary. 
(For elementary school children a lot of instruction goes on about the place-value system which underlies the Hindu-Arabic numeral system we all use.)

\begin{example}
Have you ever played with a toddler?
One fun and educational activity is to put the child on your lap and count the child's toes.
This helps the child learn about the numbers \emph{one} through \emph{ten}, but also about the matching process.
As you count, you touch the child's toes individually, and thus embody the matching.
\end{example}

From our more advanced perspective, what is happening?
Recall that $\mathbb{N}$ denotes the set of natural numbers $\mathbb{N} =\{1, 2, 3, \ldots\}$.
\begin{definition}
Fix a natural number $n$. 
The \emph{initial segement} of $\mathbb{N}$ of length $n$ is the set of all natural numbers $k$ which are no greater than $n$.
We shall denote the initial segment of length $n$ by the symbol $\lfloor n \rfloor$.
\end{definition}

For example, the initial segment $\lfloor 10 \rfloor$ is the set
\[
\lfloor 10 \rfloor = \{1, 2, 3, 4, 5, 6, 7, 8, 9, 10\}.
\]

\begin{observation}
Counting the elements of a set $X$ is the same thing as making a matching between the elements of $X$ and the elements of some initial segment $\lfloor n \rfloor$ of the natural numbers.
The number $n$ is what we usually call the number of elements of $X$.
\end{observation}


\subsection*{The Importance of Ordering}

So, how can we use this to further our goals?
The idea lies in this exercise:
\begin{example}
How many one letter mathematical words are there?

More carefully, let $W_1$ be the set of all mathematical words on the usual alphabet of uppercase letters.
How many elements does $W_1$ have?

You probably know that there are $26$ letters. 
But if you didn't know, you would make a list of all of the uppercase letters. 
Out of habit, I bet you would put them in the usual order, too.
\[
W_1 = \{ A, B, C, D, E, F, G, H, I, J, K, L, M, N, O, P, Q, R, S, T, U, V, W, X, Y, Z\}
\]
This particular list of elements, with its ordering, makes a matching of the elements of $W_1$ with the elements of $\lfloor 26 \rfloor$.

The ordering is critical here!
For example, what is the 16th element?
It is $P$.
This means that $P$ in $W_1$ is paired up with $16$ in $\lfloor 26 \rfloor$.
\end{example}

\begin{observation}
Making a list of the elements of some set is exactly the same thing as constructing a matching between the elements of that set and some initial subset of the natural numbers.

The matching rule is implicit in the list in the following way:
\emph{The ordering of the elements of the list tells us how to match them with numbers.}
The first element in the list gets matched with $1$, the second element in the list gets matched with $2$, and so on.
\end{observation}

Now, you try using this idea.
\begin{exercise}
Let $W_2$ be the set of all mathematical words with two letters chosen from the alphabet of uppercase letters.
Find a way to make a list of all the elements of $W_2$. 
Use your list to explain why there is a matching between the elements of $W_2$ and the elements of $\lfloor 676 \rfloor$.
(Note that $676 = 26\times 26$.)
\end{exercise}

To recap, the key ideas are these:
\begin{itemize}
\item Counting is really an example of a matching process. You match the elements of the set you wish to count with some special subset of $\mathbb{N}$ called an initial segment.
\item A matching with an initial segment of the natural numbers is the same thing as making a list of the elements, because the ordering of the list holds the matching rule.
\end{itemize}

\subsection*{Challenges}

To keep sharp, think about how the ideas we have learned can help you solve these problems.

\begin{exercise}
Make a list that shows how the set of all arrangements of the digits $1$, $2$, $3$ and $4$ has a matching with the set $\lfloor 24 \rfloor$.
\end{exercise}

\begin{exercise}
Let $Three = \{3, 6, 9, 12, 15, 18, \ldots\}$ be the set of all natural numbers which are multiples of $3$.
Describe how there is a matching between the elements of $Three$ and the elements of $\mathbb{N}$ hiding in this description of $Three$.
\end{exercise}

%\begin{thebibliography}{9}
%\end{thebibliography}

\end{document}
%sagemathcloud={"zoom_width":100}