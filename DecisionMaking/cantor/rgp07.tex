\documentclass[12pt,letterpaper]{article}

\usepackage[utf8]{inputenc}
\usepackage[T1]{fontenc}
\usepackage{amsmath}
\usepackage{amsfonts}
\usepackage{amssymb}
\usepackage{amsthm}
\usepackage[left=2cm,right=2cm,top=2cm,bottom=2cm,headheight=22pt]{geometry}
\usepackage{fancyhdr}
\usepackage{setspace}
\usepackage{lastpage}
\usepackage{graphicx}
\usepackage{caption}
\usepackage{subcaption}
\usepackage{paralist}
\usepackage{url}

\theoremstyle{definition}
\newtheorem{question}{Question}
\newtheorem{example}{Example}
\newtheorem{exercise}[question]{Exercise}
\newtheorem*{challenge}{Challenge}
\newtheorem*{theorem}{Theorem}
\newtheorem*{definition}{Definition}

\begin{document}

%Paramètres de mise en forme des paragraphes selon les normes françaises
\setlength{\parskip}{1ex plus 0.5ex minus 0.2ex}
\setlength{\parindent}{0pt}

%Paramètres relatifs aux en-têtes et pieds de page.
\pagestyle{fancy}
\lhead{Theron J Hitchman}
\chead{\Large Reading and Guided Practice \#7}
\rhead{Fall 2013}
\lfoot{\emph{Math and Decision Making}}
\cfoot{}
\rfoot{\emph{\thepage\ of \pageref{LastPage}}}

\section*{Introduction}
We introduce the idea of a proof by contradiction.
This is used to prove that the set of prime numbers is an infinite set.

\section*{Goals}
At the end of this assignment, a student should be able to:
\begin{compactitem}
\item State clearly what what a ``proof by contradiction'' is.
\item Describe the logic behind a proof by contradiction.
\item Explain why the set of all prime numbers is an infinite set.
\end{compactitem}


\section*{Reading and Guided Practice for 21 October}

Mathematicians have known for a long time about \emph{prime numbers}.

\begin{definition} 
A natural number $p$ which is greater than $1$ is called a \emph{prime number} when its only divisors are $1$ and $p$.
\end{definition}

\begin{example}
The first few prime numbers are 
\[2, 3, 5, 7, 11, 13, 17, 19, 23, 29, 31, \ldots\]
\end{example}

What makes prime numbers so important?
They are like little atoms for multiplication.
They are the smallest indivisible pieces of natural numbers.

Our goal today is to understand the argument for the following important fact, which has been known since at least the ancient Greek culture mathematicians. 
(This work appears in Euclid's \emph{Elements}.)

\begin{theorem}[The Infinitude of Primes]
The set $\mathbb{P} = \{ p \mid \text{$p$ is a natural number and $p$ is prime} \}$ of all prime numbers is an infinite set.
\end{theorem}

\begin{exercise}
What is the next largest prime number after $31$?
\end{exercise}

\subsection*{A Preface on the Argument}

We are going to look at the proof of that theorem.
It is not necessary to be afraid of the word \emph{proof} here.
All that we mean by the word \emph{proof} is that we are going to give a convincing argument about why the theorem is true.

But in this case, the convincing argument has a clever technique in it.
The argument is an example of a \emph{proof by contradiction}. 
What happens is this:
\begin{compactdesc}
\item[Step 1:] We begin by assuming the theorem is \textbf{false}.
\item[Step 2:] We use that assumption to argue that some other thing happens.
\item[Step 3:] The trick is that the ``other thing'' has to be obviously wrong.
\item[Step 4:] Since our assumption led us to something stupid, we conclude that our assumption must be incorrect.
\item[Step 5:] Therefore our theorem must be true.
\end{compactdesc}

This is the logic behind a proof by contradiction.
It can be easy to get things turned around a bit when you are new to this type of argument, because you have to assume negative things a lot.
People who are used to this kind of argument spend all their time on Step 2, and they do not usually explain that the other steps are happening, too.
But after a bit of practice, you will find this natural and all will be well.

\begin{exercise}
Have you ever seen a proof by contradiction before?
Try to think of one or two situations where it has come up.
\end{exercise}

\begin{exercise}
How is a proof by contradiction like a jury trial?
(This is a weak analogy, but there is something to be learned here.)
\end{exercise}

\subsection*{A Fact we need for the proof}
To make the proof run, we need a fact which you likely already know.
So that things are clearest, we state it right now up front as some thing you can and should believe.
But so we do not get distracted, we will not give a proof of this statements.

\subsubsection*{The Fundamental Theorem of Arithmetic}

\begin{theorem}
If $n$ is a natural number, then there is a unique way to write $n$ as a product of prime number factors.
\end{theorem}

There are actually two things happening here.
First, every natural number can be written as a product of prime numbers.
For example, $6=2\cdot 3$, and $100 = 2\cdot 2 \cdot 5 \cdot 5$.

\begin{exercise}
Write the number $204$ as a product of prime number factors.
\end{exercise}

Second, there is only one way to write a given number as a product of prime numbers.
The only thing you can do is rearrange the list of primes so it reads in a different order.

For example, the only interesting thing one can do with the expression for $6$ is reorder it like this: $6 = 3\cdot 2$.

\subsection*{A Proof of The Infinitude of Primes}

We shall make an argument by contradiction.
Suppose that the set $\mathbb{P}$ is not infinite.
That is, assume that $\mathbb{P}$ is a finite set.

Since $\mathbb{P}$ is finite, there must be some list of primes which stops at some point.
We write this list as a description of $\mathbb{P}$ as follows:
\[
\mathbb{P} = \{ p_1, p_2, p_3, p_4, \ldots, p_{n-1}, p_n \}.
\]
We do not know what the length $n$ is, or what each individual prime $p_i$ in our list is.
That will not matter!
What matters is that the list stops.

Here is a big leap.
Consider the new number 
\[
X = p_1\cdot p_2 \cdot \dots \cdot p_{n-1} \cdot p_n + 1,
\]
which is made by multiplying together all of the prime numbers on our list, and then adding $1$.
Note that $X$ has to be bigger than all of the numbers in our list of prime numbers.

By the Fundamental Theorem of Arithmetic, $X$ can be written as a product of prime numbers.
Which prime numbers are factors of $X$?
We shall check each of the prime numbers in our list.

Does $p_1$ divide $X$?
No.
When you try to divide $X$ by $p_1$, you will get a remainder of $1$.

Does $p_2$ divide $X$?
No.
When you try to divide $X$ by $p_2$, you will get a remainder of $1$.

This little argument works \emph{for all of the prime numbers on our list}!

Since $X$ has to have at least one prime factor, we see that this prime factor is some new, unknown number which was not on our list!
This is a problem.
All of the prime numbers are in our list.

So, something about the above is wrong.
What is wrong?
Our assumption at the beginning.

We conclude that the set $\mathbb{P}$ of all prime numbers is an infinite set.

\section{Some Exercises}

When confronted with a new argument, it can be helpful to work out several special cases to see what happens. 
This particular argument can almost be used as a recipe for trying to find new prime numbers.
When you do so, you quickly find this sequence of tasks.

\begin{exercise}
Run through the argument in the specific case where $\mathbb{P} =\{2, 3\}$.
See that the number $X$ in this case is a prime number.
\end{exercise}


\begin{exercise}
Run through the argument in the specific case where $\mathbb{P} =\{2, 3, 7\}$.
See that the number $X$ in this case is a prime number.
\end{exercise}

\begin{exercise}
Run through the argument in the specific case where $\mathbb{P} =\{2, 3, 7, 43\}$.
See that the number $X$ in this case is not prime.
What are the prime number factors of this $X$?
Are they on the list $\mathbb{P}$?
\end{exercise}

%\begin{thebibliography}{9}
%\end{thebibliography}

\end{document}
%sagemathcloud={"zoom_width":100}