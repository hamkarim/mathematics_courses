\documentclass[12pt,letterpaper]{article}

\usepackage[utf8]{inputenc}
\usepackage[T1]{fontenc}
\usepackage{amsmath}
\usepackage{amsfonts}
\usepackage{amssymb}
\usepackage{amsthm}
\usepackage[left=2cm,right=2cm,top=2cm,bottom=2cm,headheight=22pt]{geometry}
\usepackage{fancyhdr}
\usepackage{setspace}
\usepackage{lastpage}
\usepackage{graphicx}
\usepackage{caption}
\usepackage{subcaption}
\usepackage{paralist}
\usepackage{url}

\theoremstyle{definition}
\newtheorem{question}{Question}
\newtheorem{example}{Example}
\newtheorem{exercise}[question]{Exercise}
\newtheorem*{challenge}{Challenge}
\newtheorem*{theorem}{Theorem}
\newtheorem*{definition}{Definition}

\begin{document}

%Paramètres de mise en forme des paragraphes selon les normes françaises
\setlength{\parskip}{1ex plus 0.5ex minus 0.2ex}
\setlength{\parindent}{0pt}

%Paramètres relatifs aux en-têtes et pieds de page.
\pagestyle{fancy}
\lhead{Theron J Hitchman}
\chead{\Large Reading and Guided Practice \#12}
\rhead{Fall 2013}
\lfoot{\emph{Math and Decision Making}}
\cfoot{}
\rfoot{\emph{\thepage\ of \pageref{LastPage}}}

\section*{Introduction}


\section*{Goals}
At the end of this assignment, a student should be able to:
\begin{compactitem}
\item Given a rational number, identify the real number to which it corresponds.
\item Given a real number in decimal notation, decide if that number is a rational number or not.
\item Given a real number which possibly comes from a rational number, find the relevant rational number.
\end{compactitem}
A student might also be able to:
\begin{compactitem}
\item Describe a few real numbers for which the decimal notations are difficult to determine exactly.
\end{compactitem}

\section*{Reading and Questions for 30 October}

We have developed several number systems at this point, and at each stage things have gotten larger.
If you look, the natural numbers have a natural inclusion into the integers. 
(Each number $n$ basically gets sent to itself.)
Similarly, the integers have a natural inclusion into the rationals.
(An integer $a$ gets sent to the rational number $a/1$.)
In each of these inclusions, it is clear exactly which numbers come from the smaller set, and which ones do not.

But what about the rational numbers and the real numbers?
We now describe the situation.

\subsection*{Identifying a Rational Number with a Real Number}

Again, we shall focus our attention on numbers which lie between $0$ and $1$.
Suppose we are given a rational number $a/b$ in this interval.
Then we know these things:
\begin{compactitem}
\item $a$ is not negative.
\item $a \leq b$.
\end{compactitem}
What real number should this correspond to?
Well, geometrically, we divide the interval from $O$ to $I$ into $b$ equal parts, and then we take the one $a$ steps from the left.
But that does give us the decimal notation all that easily.

How do we find the decimal notation?
Suppose the decimal notation looks like 
\[
0.r_1r_2r_3r_4\ldots
\]
Then $r_1$ marks which subinterval in our system of even division into tenths.
So we must have that $r_1$ is the digit that gives us
\[
\dfrac{r_1}{10} \leq \dfrac{a}{b} < \dfrac{r_1 + 1}{10}.
\]
This is equivalent to 
\[
r_1 \leq 10\cdot \dfrac{a}{b} < r_1 + 1.
\]
This means we want to choose the biggest digit $r_1$ so that $r_1 \leq 10 \dfrac{a}{b}$.

Once we have done so, we want to look at how much is left over: 
\[
x = 10\dfrac{a}{b} - r_1
\]
This is a new rational number. 

One crucial fact is this:
\begin{quote}
\textbf{The number $x$ is a rational number, which can be expressed having the same denominator $b$ as $a/b$.
So $x = a'/b$ for some other number $a'$.}
\end{quote}

And $x$ will be smaller than $1$, because of the choices above.
The key is that we now can find $r_2$ by starting over with the number $x$.
And then you can find $r_3$ by starting over with whatever is left over from that.

Geometrically, we have replaced the ``keep subdividing into tenths'' business with the algebraic step of ``scale everything out by a factor of ten.'' This has the same effect, but is more convenient for arithmetic.


\begin{example}
Consider the rational number $5/6$.
We shall find the decimal notation for this as a real number.
\end{example}


Notice that the process has a few steps which end with ``take the left-over part and repeat.''
This means that it can keep going, with no well-defined end in sight.


\begin{exercise}
Try this basic process out with the number $7/8$.
\end{exercise}

\begin{exercise}
Try this basic process out with the number $1/6$.
\end{exercise}

Alright, now it is time for the big secret.
All of that work can be organized into an algorithm that can be performed quickly and accurately as long as you do the book-keeping properly.
That algorithm is called\dots \textbf{long division}.

\begin{exercise}
Find the decimal notation representations for $7/8$ and $17/93$.
Compare your work to what you were doing above.
Do you see how it is the same?
\end{exercise}

\subsection*{Which Real Numbers Are Rational Numbers in Disguise?}

How can we tell if one of our real numbers is a rational number?



%\begin{thebibliography}{9}
%\end{thebibliography}

\end{document}
%sagemathcloud={"zoom_width":100}