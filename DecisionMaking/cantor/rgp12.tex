\documentclass[12pt,letterpaper]{article}

\usepackage[utf8]{inputenc}
\usepackage[T1]{fontenc}
\usepackage{amsmath}
\usepackage{amsfonts}
\usepackage{amssymb}
\usepackage{amsthm}
\usepackage[left=2cm,right=2cm,top=2cm,bottom=2cm,headheight=22pt]{geometry}
\usepackage{fancyhdr}
\usepackage{setspace}
\usepackage{lastpage}
\usepackage{graphicx}
\usepackage{caption}
\usepackage{subcaption}
\usepackage{paralist}
\usepackage{url}

\theoremstyle{definition}
\newtheorem{question}{Question}
\newtheorem{example}{Example}
\newtheorem{exercise}[question]{Exercise}
\newtheorem*{challenge}{Challenge}
\newtheorem*{theorem}{Theorem}
\newtheorem*{definition}{Definition}

\begin{document}

%Paramètres de mise en forme des paragraphes selon les normes françaises
\setlength{\parskip}{1ex plus 0.5ex minus 0.2ex}
\setlength{\parindent}{0pt}

%Paramètres relatifs aux en-têtes et pieds de page.
\pagestyle{fancy}
\lhead{Theron J Hitchman}
\chead{\Large Reading and Guided Practice \#12}
\rhead{Spring 2014}
\lfoot{\emph{Math and Decision Making}}
\cfoot{}
\rfoot{\emph{\thepage\ of \pageref{LastPage}}}

\section*{Introduction}
We recap some of the different types of sets we have encountered, and try to group them by ``size."

\section*{Goals}
At the end of this assignment, a student should be able to:
\begin{compactitem}
\item Name a variety of different types of sets.
\item Explain clearly to someone else a surprising example of two sets that are really ``the same size.''
\end{compactitem}

\section*{Reading and Questions for Cantor's Paradise Meeting 13}

How many truly different sizes of sets have we encountered?
There are a couple of important distinctions made in our work so far.
Where do our various examples sit?

\subsection*{Finite Sets}

Recall that a set is \emph{finite} if its elements can be matched with the elements of some initial segment $\lfloor n \rfloor = \{ 1, 2, 3, \ldots , n \}$ of the natural numbers.

The set $\{A, B, C\}$ has three elements and is finite.
The set of all arrangements of the letters $ABC$ is also finite, though this time there are six elements.
The official roster for \emph{Math 1100-03 Math in Decision Making, in Fall 2013} is a set $M$ of students with $67$ elements.
The set
\[
C = \{ X \mid \text{$X$ is a two element subset of $M$} \}
\]
is a bigger set, but it is still finite.
In fact, $C$ has $\frac{66\cdot 67}{2} = 2211$ elements.

We have also come across the sets
\[
B_5 = \{ \text{$0/1$ strings of length $5$} \}
\]
and 
\[
\mathcal{P}(\lfloor 5 \rfloor) = \{ X \mid \text{ $X$ is a subset of $\lfloor 5 \rfloor$} \}
\] and seen that these are the same size.

In fact, that power set construction is quite powerful, and can be used to make ever larger and larger sets. In general, if $X$ has $n$ elements, then $\mathcal{P}(X)$ has $2^n$ elements.


\subsection*{Infintite Sets}

What examples of inifinite sets have we seen?
A partial list looks like this:
\[
\mathbb{N}, \mathbb{Z}, \mathbb{Q}^+, \mathbb{Q}, \mathcal{E}, \mathcal{O}, \mathcal{W}
\]
For the naturals, the integers, the positive rationals, the rationals, the evens, the odds, and finally, the set of all ``finite length mathematical words.''
These funny sets have the distinguishing characteristic that they are all the same size as $\mathbb{N}$.
Such sets are called \emph{countably infinite}.
Of course, that means that in some sense all of these sets have the same size as each other.

\begin{exercise}
Find someone who isn't in this class, but could be, and explain to them how weird and wonderful it is that the sets $\mathbb{N}$ and $\mathbb{Q}$ have the same number of elements.
\end{exercise}

But we have encountered other infinte sets, too. 
\begin{compactdesc}
\item[$\mathbb{R}$]: the real numbers,
\item[$\mathcal{R}$]: the real numbers between $0$ and $1$,
\item[$B_{\infty}$]: the set of infinitely long $0/1$ strings,
\item[$\mathcal{P}(\mathbb{N})$]: the collection of all subsets of the natural numbers, 
\item[$\mathcal{C}$]: the set of all infinte words on the letters $L$ and $R$, and
\item[$\mathcal{I}$]: the set of all infinite words on the letters $l$, $c$ and $r$,
\item[$\mathcal{I}_c$]: the set of all infinite words on the letters $l$, $c$ and $r$, but that do not use $c$.
\end{compactdesc}

\begin{exercise}
Are any of the sets in this last list provably of ``the same size?''
That is, can you find a matching between any pair of these?
Go through what you have learned so far and sort out what is what.

Which sets have we not, yet, figured out size relative to some set we ``understand?''
\end{exercise}

%\begin{thebibliography}{9}
%\end{thebibliography}

\end{document}
%sagemathcloud={"zoom_width":100}