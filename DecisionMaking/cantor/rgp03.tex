\documentclass[12pt,letterpaper]{article}

\usepackage[utf8]{inputenc}
\usepackage[T1]{fontenc}
\usepackage{amsmath}
\usepackage{amsfonts}
\usepackage{amssymb}
\usepackage{amsthm}
\usepackage[left=2cm,right=2cm,top=2cm,bottom=2cm,headheight=22pt]{geometry}
\usepackage{fancyhdr}
\usepackage{setspace}
\usepackage{lastpage}
\usepackage{graphicx}
\usepackage{caption}
\usepackage{subcaption}
\usepackage{paralist}
\usepackage{url}

\theoremstyle{definition}
\newtheorem{question}{Question}
\newtheorem{example}{Example}
\newtheorem{exercise}[question]{Exercise}
\newtheorem*{challenge}{Challenge}
\newtheorem*{theorem}{Theorem}
\newtheorem*{definition}{Definition}

\begin{document}

%Paramètres de mise en forme des paragraphes selon les normes françaises
\setlength{\parskip}{1ex plus 0.5ex minus 0.2ex}
\setlength{\parindent}{0pt}

%Paramètres relatifs aux en-têtes et pieds de page.
\pagestyle{fancy}
\lhead{Theron J Hitchman}
\chead{\Large Reading and Guided Practice \#3}
\rhead{Spring 2014}
\lfoot{\emph{Math and Decision Making}}
\cfoot{}
\rfoot{\emph{\thepage\ of \pageref{LastPage}}}

\section*{Introduction}
In this reading, we learn about the idea of a matching. 
This is a way to compare the sizes of two sets without actually counting them.

\section*{Goals}
At the end of this assignment, a student should be able to:
\begin{compactitem}
\item Describe the idea of a matching between two sets.
\item Use the idea of a matching between two finite sets to show that those sets have the same number of elements.
\item recognize a common pitfall in constructing matchings.
\end{compactitem}
A student might also be able to:
\begin{compactitem}
\item Solve a challenging problem using a matching between sets.
\end{compactitem}

\section*{Reading and Questions for Cantor's Paradise Meeting 4}

Mathematicians want to understand the relative sizes of very large sets.
But it can be difficult to count the number of elements of a large set.
Worse, it could be inconvenient.
But there is a way to keep track of the idea of two sets ``having the same size'' without actually knowing what the size is!
This is our goal for today.

\subsection*{The Idea of a Matching: A Parable?}
Have you ever tried counting to a big number really fast?
It gets challenging as the words pile up. 
\emph{One hundred seventy-seven} is a lot of syllables.

Imagine a shepherd with a large flock.
The sheep move in and out of the pen a bit too fast for him to count, as the words describing the numbers get to be a mouthful. 
Instead, he gets a bag of pebbles and makes sure he has one pebble for each sheep. 
How many pebbles is it?
It doesn't matter.
What matters is that he can match them up with the sheep.
Each morning, he moves one pebble for each sheep that passes out of the gate into a second bag.
If the last sheep goes out as the last pebble switches bags, he has seen them all.
This can be repeated in the evening. 
If the last sheep comes in as the last pebble changes from one bag to the other, then all of the sheep have come home.
If the sheep and the pebbles do not match, he knows if any sheep are missing, or if there is somehow a new addition to the flock.

This is the idea of a matching.
In this case, the flock of sheep is one set (whose elements are the individual sheep), and the bag of pebbles is the other set (whose elements are the individual pebbles).
Each morning and each evening, the farmer is observing a matching between the elements of these two sets.

\begin{definition}
Let $A$ and $B$ be two sets. A \emph{matching between the elements of $A$ and the elements of $B$} is a way to associate to each element of $A$ exactly one element of $B$, so that no elements of either set are left unpaired.
\end{definition}

\begin{example}
Let $A$ be the set of fingers on your right hand. 
Let $B$ be the set of fingers on your left hand. 
It is usually the case that these two sets have the same size. 
The most common matching between the elements of $A$ and the elements of $B$ is made by pressing your fingertips together like Mr. Burns from \emph{The Simpsons}. 
\textbf{Excellent!}
\end{example}

\begin{exercise}
Let $T$ be the set of all arrangements of the letters $ABCDEFGHIJ$.
Let $J$ be the set of all arrangements of the letters $KLMNOPQRST$.
First, be sure to figure out what counts as an element of $T$ and what counts as an element of $J$.
(Sanity Check: There are lots of elements of both sets.
Way more than 100.)

Find a matching between the elements of $T$ and the elements of $J$.
Use this to describe how you know that these two sets have the same size, even though you don't know what that size is.
\end{exercise}


\begin{exercise}
Let $T = \{ 1, 4, 7, 10, 13, \ldots, 3001\}$.
Let $J = \{ 2, 5, 8, 11, 14, \ldots, 3002\}$.
Describe a matching between the elements of $T$ and the elements of $J$.

As a check: can you see why these two sets both have $1001$ elements?
\end{exercise}

\subsection*{One Common Pitfall}

Even after you get the basic idea of a matching, it pays to be careful. 
There are a few common mistakes one can make. 
One of them is encapsulated in the next exercise.

\begin{exercise}
My friend Penny is very young, and is still learning to count.
When she first tried it, she often counted things like this:
\begin{quote}
One, Three, Four, Five, Eight, \dots
\end{quote}
What is the nature of Penny's mistake?

Can you use the language of matchings to say exactly what Penny's mistake is?
\end{exercise}

\subsection*{A Challenge}

\begin{challenge}
Horror of horrors, you are in charge of every third grader in Cedar Falls for an afternoon.
Fortunately, you have them all trapped in the UNI-Dome.
They are all running around on the field shouting, "Look at me! I'm a vampire unicorn!"

Quick, without counting, come up with a way to decide if there are more boys, more girls, or the same number of each.
\end{challenge}

%\begin{thebibliography}{9}
%\end{thebibliography}

\end{document}
%sagemathcloud={"zoom_width":100}