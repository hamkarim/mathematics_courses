\documentclass[12pt,letterpaper]{article}

\usepackage[utf8]{inputenc}
\usepackage[T1]{fontenc}
\usepackage{amsmath}
\usepackage{amsfonts}
\usepackage{amssymb}
\usepackage{amsthm}
\usepackage[left=2cm,right=2cm,top=2cm,bottom=2cm,headheight=22pt]{geometry}
\usepackage{fancyhdr}
\usepackage{setspace}
\usepackage{lastpage}
\usepackage{graphicx}
\usepackage{caption}
\usepackage{subcaption}
\usepackage{paralist}
\usepackage{url}

\theoremstyle{definition}
\newtheorem{question}{Question}
\newtheorem{example}{Example}
\newtheorem{exercise}[question]{Exercise}
\newtheorem*{challenge}{Challenge}
\newtheorem*{theorem}{Theorem}
\newtheorem*{definition}{Definition}
\newtheorem*{observation}{Observation}

\begin{document}

%Paramètres de mise en forme des paragraphes selon les normes françaises
\setlength{\parskip}{1ex plus 0.5ex minus 0.2ex}
\setlength{\parindent}{0pt}

%Paramètres relatifs aux en-têtes et pieds de page.
\pagestyle{fancy}
\lhead{Theron J Hitchman}
\chead{\Large Reading and Guided Practice \#11}
\rhead{Fall 2013}
\lfoot{\emph{Math and Decision Making}}
\cfoot{}
\rfoot{\emph{\thepage\ of \pageref{LastPage}}}

\section*{Introduction}
We discuss the relationship between rational numbers and real numbers.

\section*{Goals}
At the end of this assignment, a student should be able to:
\begin{compactitem}
\item Given a rational number, identify the real number to which it corresponds.
\item Given a real number in decimal notation, decide if that number is a rational number or not.
\item Given a real number which possibly comes from a rational number, find the relevant rational number.
\end{compactitem}
A student might also be able to:
\begin{compactitem}
\item Describe a few real numbers for which the decimal notations are difficult to determine exactly.
\end{compactitem}

\section*{Reading and Questions for 30 October}

We have developed several number systems at this point, and at each stage things have gotten larger.
If you look, the natural numbers have a natural inclusion into the integers. 
(Each number $n$ basically gets sent to itself.)
Similarly, the integers have a natural inclusion into the rationals.
(An integer $a$ gets sent to the rational number $a/1$.)
In each of these inclusions, it is clear exactly which numbers come from the smaller set, and which ones do not.

But what about the rational numbers and the real numbers?
We now describe the situation.

\subsection*{Identifying a Rational Number with a Real Number}

Again, we shall focus our attention on numbers which lie between $0$ and $1$.
Suppose we are given a rational number $a/b$ in this interval.
Then we know these things:
\begin{compactitem}
\item $a$ is not negative.
\item $a \leq b$.
\end{compactitem}
What real number should this correspond to?
Well, geometrically, we divide the interval from $O$ to $I$ into $b$ equal parts, and then we take the one $a$ steps from the left.
But that does give us the decimal notation all that easily.

How do we find the decimal notation?
Suppose the decimal notation looks like 
\[
0.r_1r_2r_3r_4\ldots
\]
Then $r_1$ marks the subinterval containing $a/b$ in our system of even division into tenths.
So we must have that $r_1$ is the digit that gives us
\[
\dfrac{r_1}{10} \leq \dfrac{a}{b} < \dfrac{r_1 + 1}{10}.
\]
This is equivalent to 
\[
r_1 \leq 10\cdot \dfrac{a}{b} < r_1 + 1.
\]
This means we want to choose the biggest digit $r_1$ so that $r_1 \leq 10 \cdot\dfrac{a}{b}$.

Once we have done so, we want to look at how much is left over: 
\[
x = 10\dfrac{a}{b} - r_1
\]
This is a new rational number. 

One crucial fact is this:
\begin{quote}
\textbf{The number $x$ is a rational number, which can be expressed having the same denominator $b$ as $a/b$.
So $x = a'/b$ for some other number $a'$.}
\end{quote}

And $x$ will be smaller than $1$, because of the choices above.
The key is that we now can find $r_2$ by starting over with the number $x$.
And then you can find $r_3$ by starting over with whatever is left over from that.

Geometrically, we have replaced the ``keep subdividing into tenths'' business with the algebraic step of ``scale everything out by a factor of ten.'' This has the same effect, but is more convenient for arithmetic.

Notice that the process has a few steps which end with ``take the left-over part and repeat.''
This means that it can keep going, with no well-defined end in sight.

\begin{example}
Consider the rational number $5/6$.
We shall find the decimal notation for this as a real number.

First, we note that $10\cdot\frac{5}{6} = \frac{50}{6}$ lies between $8$ and $9$, since $8\cdot 6 = 48$.
So $r_1 = 8$ and $x_1 = \frac{50}{6} - \frac{48}{6} = \frac{2}{6}$.
We will start over with the new number $2/6$.
Note: We will not reduce this fraction.
Things will be clearer for our ``big picture'' if we leave it alone.

Step Two: note that $10 \cdot \frac{2}{6} = \frac{20}{6}$ lies between $3$ and $4$, since $3\cdot 6 = 18$.
So $r_2 = 3$ and $x_2 = \frac{20}{6} - \frac{18}{6} = \frac{2}{6}$.

Hey, wait! 
That is the same as before!
This is now stuck and will keep repeating the value $3$ for the digits $r_i$.

We conclude that $\frac{5}{6} = 0.8333333\ldots$
Writing a bunch of repeating threes is boring.
Instead, let's put a bar over one three, and agree the bar means ``repeat this stuff over and over forever.''
\[
\frac{5}{6} = 0.8\overline{3}.
\]
We have now completely understood this one case.
\end{example}


\begin{exercise}
Try this basic process out with the number $7/8$.
\end{exercise}

\begin{exercise}
Try this basic process out with the number $1/6$.
\end{exercise}

Alright, now it is time for the big secret.
All of that work can be organized into an algorithm that can be performed quickly and accurately as long as you do the book-keeping properly.
That algorithm is called\dots \textbf{long division}.

\begin{exercise}
Find the decimal notation representations for $7/8$ and $17/93$.
Compare your work to what you were doing above.
Do you see how it is the same?
\end{exercise}

\subsection*{Which Real Numbers Are Rational Numbers in Disguise?}

How can we tell if one of our real numbers is a rational number?
The clearest way to figure this out is to observe closely what the process is, and remember not to change the denominators.

Let us begin with a rational number and see what is possible.
If our input to the long division process is $a/b$, then at each step compute some value $r_i$ and we replace the current problem with a new one, where the new input has the form $a'/b$. 
We repeat this over and over, and there are only $b$ different options for values of the numerator, because the numerators come from this list:
\[
0, 1, 2, 3, \ldots, b-2, b-1
\]
So, somewhere along the line, but definitely by the time we have done the basic process $b$ different times, we will get a repeat of a numerator we have seen before.
At that point, we get that the whole process starts repeating.

Note: if you ever get $x = 0/b$, then you just get $0$'s from then on.
$0$'s make life easier.
In fact, you are probably used to just quitting after you get the first $0$.
It seems weird to write $1/2 = 0.5000000000\ldots$ \emph{and keep writing all of the zeros}.
Usually, people just stop and say that the decimal notation has \emph{terminated}.
By our original definition, the notation has all of those zeros in it!
So $0$ is a bit of a special case.
But it is possible to have the numerators bounce around to all of the other $b-1$ choices before you get a repeat.

So, we just observe this:

\begin{observation}
If a real number is an expression of a rational number then its decimal notation is \emph{eventually repeating}.
That means that after some initial block of dancing around, we get some other block of digits which will repeat.
Furthermore, the block of repeating digits of a number $a/b$ is no longer than $b-1$ digits long.
\end{observation}

Of course, the other direction is true, too. If the decimal notation for a number is eventually repeating, then that real number is a rational number.

The way to see this is to use a clever algebraic trick, which we now illustrate on the real number 
\[
y = 0.73\overline{25} = 0.732525252525252525\ldots
\]

\begin{example}
First, multiply by $100$ to move the decimal point two places.
Then subtract out the part which does not repeat to focus our attention on the repeating bit.
We shall just consider the number $z$ given by 
\[
z = 100 \cdot y - 73 = 0.\overline{25} = 0.2525252525\ldots
\]
We have done clean up to remove the non-repeating part for now.
We will go back and take care of it in a minute.
If we perform a similar trick with $z$, we can make this neat little bit of magic happen:
\[
100\cdot z -25 = 0.\overline{25} = z
\]
Hey, $z$ fits in a linear equation!
We can solve that to find $z = \frac{25}{99}$.

Finally, we put this information back in our expression for $y$
\[
\frac{25}{99} = 100 \cdot y -73
\]
and solve for $y$
\[
y = \frac{1}{100} \left( 73 + \frac{25}{99} \right) = \frac{73 \cdot 99 + 25}{9900} = \frac{7252}{9900} .
\]
That is clearly a rational number.
If you are more comfortable, you can put this rational number into lowest terms 
\[
y = \frac{1813}{2475}.
\]
\end{example}

\begin{exercise}
Using the above example as a guide, turn the following eventually repeating decimal notations into the standard form for rational numbers.
\begin{compactitem}
\item $x = 0.47\overline{2}$.  \hspace{.5in} (\emph{Hint: the answer is $17/36$}.)
\item $x = 0.\overline{428571}$
\end{compactitem}
\end{exercise}


\subsection*{What about Real Numbers which are not Rational Numbers}

It is convention to refer to a real number which is not a rational number as an \emph{irrational number}. 
It is not a very nice name.

So, above we described which real numbers are rational numbers by using their decimal notations.
What does this mean for the decimal notations of irrational numbers?

\begin{challenge}
What can you say about the decimal notation of the following numbers, which are known to be irrational?
What property to they have in common?
\[
\pi, \quad \sqrt{2}, \quad \sqrt{3}, \quad e
\]
\end{challenge}


\end{document}
%sagemathcloud={"zoom_width":100}