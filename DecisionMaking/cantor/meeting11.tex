\documentclass[12pt]{amsart}
\usepackage[margin=1in]{geometry}

\theoremstyle{definition}
\newtheorem{task}{Task}

\begin{document}

\begin{center}
\textbf{\Huge
Cantor's Paradise: Class Meeting \#11
}
\end{center}

\vspace{.5in}

\begin{task}
The set $\mathbb{N}\times\mathbb{N}$ is the set of all ordered pairs of natural numbers. For example, $(1,2)$, $(3,1)$, $(9,9)$ and $(403, 173455)$ are all elements of $\mathbb{N}\times\mathbb{N}$.

Prove that $\mathbb{N}\times\mathbb{N}$ is countably infinite.
\end{task}

\begin{task} Recall that $\mathcal{W}$ is the set of all mathematical `words.'
Each word is a finite length string of symbols, where each symbol is allowed to be an uppercase letter from the regular alphabet used in English.
Again, a word doesn't have to be a pronounceable English word.
It might be nonsense.

Prove that $\mathcal{W}$ is countably infinite.
\end{task}

\hrulefill
\vspace{0.5in}

Up until now, our sets of "mathematical words" only contained words of finite length. That is, $\mathcal{W}$ has a lot of elements, but each word has to stop at some point. A word in $\mathcal{W}$ can be really long, but can't go on forever.

Now we are going to change that. The set $\mathcal{C}$ consists of all infinite length mathematical words made up of the two uppercase letters L and R. Here are the \emph{beginnings} of some words in $\mathcal{C}$.

\begin{center}
\begin{tabular}{c}
LLLLLLLLLLLLLLLLLLLL \dots \\[.2cm]
LRLRLRLRLRLRLRLRLRL \dots \\[.2cm]
RRRRRRRRRRRRRRRRR \dots \\[.2cm]
LRLRRRLRRRRLRRRRRL \dots \\[.2cm]
LRLRLLRRLLRLRRRRLRL \dots\\[.5cm]
\end{tabular}
\end{center}
Similarly, we consider a set $\mathcal{I}$ which is made up of all infinite length mathematical words made up of the three lowercase letters l, c and r. For example, here are the \emph{beginnings} of three words in the set $\mathcal{I}$:

\begin{center}
\begin{tabular}{c}
llllllclrlllclrlclll \dots \\[.2cm]
cclrrrrrrrrrlclrl \dots \\[.2cm]
ccccccccclccccc \dots \\[.5cm]
\end{tabular}
\end{center}


\begin{task}
The set $\mathcal{I}_c$ is the subset of all elements of $\mathcal{I}$ which don't ever use the letter c. Design a matching between the elements of $\mathcal{C}$ and the elements of $\mathcal{I}_c$.
\end{task}



\end{document}
%sagemathcloud={"zoom_width":100}