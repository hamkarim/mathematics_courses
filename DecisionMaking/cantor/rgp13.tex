\documentclass[12pt,letterpaper]{article}

\usepackage[utf8]{inputenc}
\usepackage[T1]{fontenc}
\usepackage{amsmath}
\usepackage{amsfonts}
\usepackage{amssymb}
\usepackage{amsthm}
\usepackage[left=2cm,right=2cm,top=2cm,bottom=2cm,headheight=22pt]{geometry}
\usepackage{fancyhdr}
\usepackage{setspace}
\usepackage{lastpage}
\usepackage{graphicx}
\usepackage{caption}
\usepackage{subcaption}
\usepackage{paralist}
\usepackage{url}

\theoremstyle{definition}
\newtheorem{question}{Question}
\newtheorem{example}{Example}
\newtheorem{exercise}[question]{Exercise}
\newtheorem*{challenge}{Challenge}
\newtheorem*{theorem}{Theorem}
\newtheorem*{definition}{Definition}

\begin{document}

%Paramètres de mise en forme des paragraphes selon les normes françaises
\setlength{\parskip}{1ex plus 0.5ex minus 0.2ex}
\setlength{\parindent}{0pt}

%Paramètres relatifs aux en-têtes et pieds de page.
\pagestyle{fancy}
\lhead{Theron J Hitchman}
\chead{\Large Reading and Guided Practice \#13}
\rhead{Fall 2013}
\lfoot{\emph{Math and Decision Making}}
\cfoot{}
\rfoot{\emph{\thepage\ of \pageref{LastPage}}}

\section*{Introduction}
We introduce another way to describe real numbers (a version of \emph{binary notation}).
This involves making infintely long strings of only two symbols.

\section*{Goals}
At the end of this assignment, a student should be able to:
\begin{compactitem}
\item Find the binary notation for a real number in the set $\mathcal{R}$.
\item Given the binary notation for a real number, locate the element of $\mathcal{R}$.
\end{compactitem}
Also, a student may be able to:
\begin{compactitem}
\item Describe the possiblities for confusion when using binary notation.
\end{compactitem}

\section*{Reading and Questions for 4 November}

Our principal way of understanding real numbers uses decimal notation.
We build decimal notation (as a system) by successively dividing intervals into equal tenths.
We divide our given unit interval $OI$ into $10$ equal subintervals, then we divide each of those intervals into ten equal subintervals, then we divide each of \emph{those} intervals into $10$ equal subintervals, and so on.


Ther is not anything special about the number $10$ here.
(Well, humans generally have $10$ fingers.)
We can set up the same kind of system using other numbers as a base.

\subsection*{Binary Notation}

We begin with the portion of sime line between two marked points $O$ and $I$.
Like before, we think of these points as representing the real numbers between $0$ and $1$.

Divide this interval into two subintervals of equal size.
We shall label these new intervals $L$ and $R$, for ``left'' and ``right.''

\begin{exercise}
Draw this picture and label the subintervals.
\end{exercise}

Points in the left-hand interval will have binary notation starting with the symbol $L$.
Points in the right-hand interval will have binary notation starting with the symbol $R$.

Next we further divide these intervals into two more subintervals each.
These new subintervals get labels $LL$, $LR$, $RL$, and $RR$, reading from left to right.

\begin{exercise}
Draw the diagram showing both the two subintervals at level one and the four intervals at level two.
Label all of these intervals clearly.
\end{exercise}

This process of subdivision continues on for as long as we can stomach it.
(Well, longer, really.)

\begin{exercise}
Make a much larger diagram, and label all of the first \textbf{three} levels of subdivision in this $L$/$R$ system.
(Perhaps you want a fresh piece of paper?)
\end{exercise}

Just as a decimal notation keeps going, making ever longer and longer strings of digits, binary notation is an infinite string of the symbols $L$ and $R$.
Any initial finite block of symbols represents a particular subinterval where the point must lie.
To find the next symbol in the notation, you divide the current subinterval into two equal pieces and pick out $L$ if the point lies in the left hand one or $R$ if the point lies in the right-hand one.

\begin{exercise}
The symbol $LRRL$ corresponds to some subinterval.
Explain why this interval has length $1/16$ that of the original unit interval.
What are the endpoints of this interval?
(Try to express the endpoints as rational numbers.)
\end{exercise}

\begin{exercise}
The symbol $RLRRRRLR$ corresponds to some subinterval.
What is the length of this interval?
What are the endpoints of this interval?
(Try to express the endpoints as rational numbers.)
\end{exercise}

\begin{exercise}
Find the first five symbols in the binary notation for the rational number $1/3$.
\end{exercise}

\begin{exercise}
Find the first five symbols in the binary notation for a real number whose decimal notation starts
$0.2433\ldots$
\end{exercise}


\subsection*{Challenge}

When we studied decimal notation, we saw that it is possible for a given real number to have \emph{two} decimal notations.
For example, $0.24999999999\ldots$ is really the same point as $0.25000000000\ldots$

A similar ambiguity can happen with binary notation!

\begin{challenge}
Describe the way in which a real number might have two binary notations.
Be as specific as you can.

Can you give a concrete example?
\end{challenge}



%\begin{thebibliography}{9}
%\end{thebibliography}

\end{document}
%sagemathcloud={"zoom_width":100}