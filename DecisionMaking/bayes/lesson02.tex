\documentclass[12pt]{amsart}
\usepackage[margin=1in]{geometry}
\usepackage{paralist}

\theoremstyle{definition}
\newtheorem{question}{Question}

\begin{document}
\begin{center}
\textbf{\Huge
Lesson Plan: Meeting Two
}
\end{center}
\vspace{.5in}

Do Peer Instruction Polling using terms from basic probability.

\begin{question}
Let's play a dice rolling game. Each side of my die is painted black or red. Sides usually marked 1, 2, 3, and 4 are painted black. Sides usually marked 5 and 6 are painted red. Black--I win; red--you win. Is this game fair?
\end{question}

\begin{question}
It is possible to learn to flip a coin so that if heads is up when you flip, tails will be up when you catch, and if tails is up when you flip, heads will be up when you catch. Is betting on this kind of coin flip fair?
\end{question}


\begin{question}
To generate some numbers, let's agree to measure the birth weights of newborns in Iowa, in pounds and ounces like usual, and take only the last digit of the pounds part. What is the sample space here?
\end{question}


\begin{question}
Keeping with the newborns as our random number generator: Is this setup biased or unbiased?
\end{question}


\begin{question}
Suppose our game is to throw two ``fair'' dice and take the sum of the numbers that show. Which should you expect?
\begin{compactitem}
\item To throw 7 more frequently than 6.
\item To throw 7 less frequently than 6. 
\item To throw 7 and 6 equally often.
\end{compactitem}
\end{question}


\end{document}