\documentclass[12pt]{amsart}
\usepackage[margin=1in]{geometry}
\usepackage{paralist}

\theoremstyle{definition}
\newtheorem{task}{Task}

\begin{document}

\begin{center}
\textbf{\Huge
Probability: Class Meeting \# 3
}
\end{center}
\vspace{.5in}

I encourage you to think about setting up the sample spaces carefully in these situations.
Make diagrams that might help.

\subsection*{One Card}
A card is drawn from a standard deck of playing cards which have been well-shuffled seven times.
What is the probability that the card is:
\begin{compactitem}
\item Either a face card (jack, queen, king) or a ten?
\item Either a spade or a face card?\\
\end{compactitem}

\subsection*{Find the Mistake}
Consider a situation where the random trial consists of rolling one fair six-sided die and recording the number shown on the top.
Let $E$ be the event that an even number is up. Let $M$ be the event that either 1, 2, 3, or 5 is up.\\

David says that the probability of the event $E \cup M$ (read as ``$E$ or $M$") is 7/6. What was his reasoning, and where did he go wrong?

\subsection*{Another ``or''}
Consider a game of chance where we roll a six-sided die and pull a single card out of a regular deck of 52 playing cards.
That is a single trial consists of doing both things.
What is the probability of either rolling a $1$ or picking a $\heartsuit$.

\vspace{.5in}
\hrule
\vspace{.25in}

Now, not every probability situation can be modeled with the word ``or.''
Try your had at this set-up.

\subsection*{Archery}
We will denote the probability of an event $A$ with notation $\Pr(A)$.

An archer's target has four concentric circles around a bull's-eye. For a certain archer, the probabilities of scoring are as follows:
\begin{compactitem}
\item $\Pr(\text{hit the bull's-eye}) = 0.1$.
\item $\Pr(\text{hit first circle, but not bull's-eye}) = 0.3$.
\item $\Pr(\text{hit second circle, but no better}) = 0.2$.
\item $\Pr(\text{hit third circle, but no better}) = 0.2$.
\item $\Pr(\text{hit fourth circle, but no better}) = 0.1$.
\end{compactitem}
We will assume that each shot she takes has nothing to do with the outcomes of the other shots she takes, so they are all dealt with separately.
\begin{compactitem}
\item[a)] What is the probability that in two shots she scores a bull's-eye on the first shot, and the third circle on her second shot?
\item[b)] What is the probability that in two shots she hits the bull's-eye once and the third circle once.
\item[c)] What is the probability that on any one shot she misses the target entirely?
\end{compactitem}

\end{document}
%sagemathcloud={"zoom_width":100}