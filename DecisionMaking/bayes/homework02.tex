%% Fall 2013 MDM Homework Template
\documentclass[12pt,letterpaper]{article}

\usepackage[utf8]{inputenc}
\usepackage[T1]{fontenc}
\usepackage{amsmath}
\usepackage{amsfonts}
\usepackage{amssymb}
\usepackage[left=2cm,right=2cm,top=2cm,bottom=2cm,headheight=22pt]{geometry}
\usepackage{fancyhdr}
\usepackage{setspace}
\usepackage{lastpage}
\usepackage{graphicx,subcaption}
\usepackage{paralist}

\begin{document}

%other parameters
\setlength{\parskip}{1ex plus 0.5ex minus 0.2ex}
\setlength{\parindent}{0pt}

%header and footer parameters
\pagestyle{fancy}
\lhead{Math 1100 section 03}
\chead{Weekly Homework}
\rhead{Due: date here}
\lfoot{}
\cfoot{\emph{Prof. Hitchman}}
\rfoot{}

\begin{center}
{
\Large
\textbf{Written Assignment \#2}
}
\end{center}

\subsection*{The Gambler's Fallacy} A common (and easy to make) logical error is called \emph{The Gambler's Fallacy}. It goes a bit like this:

A gambler is betting on what he believes is a \emph{fair} roulette wheel. The wheel is divided into 38 segments, of which 
\begin{compactitem}
\item 18 are black,
\item 18 are red, and
\item 2 segments are green, and marked with zeros.
\end{compactitem}

If you bet \$ 10 on red and the wheel stops on red, you win \$ 20. Likewise with black. Otherwise you lose. The house always wins when the wheel stops at zero.

Now imagine that there has been a long run---a dozen spins---in which the wheel stopped at black. The gambler decides to bet on red, because he thinks:

\begin{quote} The wheel must come up red soon.\\
This wheel is fair, so it stops on red as often as it stops on black.\\
Since it has not stopped on red recently, it must stop there soon. I'll bet on red.
\end{quote}

Do you agree? I'll give you a hint: There is an error\dots 

\subsection*{The Dry August} The set-up is like this:
\begin{quotation}
A man started a farm in arid but fertile land. 
Weather records show that during each month from March through August there is a fifty-fifty chance of one good rainfall, enough to let the man irrigate for the whole growing season. 
A completely dry growing season occurs less than once in sixty years.

It is now mid-August. 
There has been no rain since a big storm at the end of last year. 
The farmer has run out of water. 
His crops are dying. 
He is optimistic, because he thinks:
\begin{center}
Long stretches without rain are pretty rare. 
It will almost certainly rain soon. 
The statistics prove it!
\end{center}
\end{quotation}
Again, there is an error in the reasoning here.

\subsection*{What to do:}

These two situations have some faulty reasoning.
Explain the troubles. 
How are these situations alike? 
How are they different?
What has gone wrong in the reasoning made by the gambler? The farmer?
Be sure to include any diagrams you make as you figure things out.



\end{document}
%sagemathcloud={"zoom_width":100}