\documentclass[12pt,letterpaper]{article}

\usepackage[utf8]{inputenc}
\usepackage[T1]{fontenc}
\usepackage{amsmath}
\usepackage{amsfonts}
\usepackage{amssymb}
\usepackage{amsthm}
\usepackage[left=2cm,right=2cm,top=2cm,bottom=2cm,headheight=22pt]{geometry}
\usepackage{fancyhdr}
\usepackage{setspace}
\usepackage{lastpage}
\usepackage{graphicx}
\usepackage{caption}
\usepackage{subcaption}
\usepackage{paralist}
\usepackage{url}

\theoremstyle{definition}
\newtheorem{question}{Question}
\newtheorem{example}{Example}
\newtheorem{exercise}[question]{Exercise}
\newtheorem*{challenge}{Challenge}
\newtheorem*{theorem}{Theorem}
\newtheorem*{definition}{Definition}

\begin{document}

%Paramètres de mise en forme des paragraphes selon les normes françaises
\setlength{\parskip}{1ex plus 0.5ex minus 0.2ex}
\setlength{\parindent}{0pt}

%Paramètres relatifs aux en-têtes et pieds de page.
\pagestyle{fancy}
\lhead{Theron J Hitchman}
\chead{\Large Reading and Guided Practice \#2}
\rhead{Fall 2013}
\lfoot{\emph{Math and Decision Making}}
\cfoot{}
\rfoot{\emph{\thepage\ of \pageref{LastPage}}}

\section*{Introduction}
We introduce our idea of probability as a measurement, and we discuss some basic ways to compute probability of events.

\section*{Goals}
At the end of this assignment, a student should be able to:
\begin{compactitem}
\item State clearly the frequentist interpretation of probability
\item Compute the probability of some simple events.
\item Recognize situations where ``disjointness'' applies, and appropriately use some rules to compute probability in these situations.
\end{compactitem}

\section*{Reading and Questions for 15 November}

\subsection*{An Example}

Consider a game where we roll two fair dice. 
These dice are the usual standard gaming dice, with six sides labeled with the numbers 1 through 6. 
(If you like the traditional kind, then the dice do not have number labels, but instead have the corresponding number of little dents. 
Those dents are called \emph{pips}. 
That is not really important, but it is kind of neat how there are special words for lots of crazy things.)

The outcomes in this situation are the pairs of numbers that show up when then the dice stop rolling.
We can then represent the sample space here as the collection of all possible pairs of numbers.
The idea behind this is to remember that there are two dice!
Moving horizontally changes the value on one die, but leaves the other alone.
If we switch the roles of the dice, the same is true for moving vertically.
\begin{center}
\begin{tabular}{cccccc}
1,1 & 1,2 & 1,3 & 1,4 & 1,5 & 1,6\\
2,1 & 2,2 & 2,3 & 2,4 & 2,5 & 2,6\\
3,1 & 3,2 & 3,3 & 3,4 & 3,5 & 3,6\\
4,1 & 4,2 & 4,3 & 4,4 & 4,5 & 4,6\\
5,1 & 5,2 & 5,3 & 5,4 & 5,5 & 5,6\\
6,1 & 6,2 & 6,3 & 6,4 & 6,5 & 6,6
\end{tabular}
\end{center}

Notice that the sample space has $36$ different elements in it.
Each of these outcomes is equally like as any other, so this is an unbiased situation.

\subsection*{The Frequentist Interpretation of Probability}

In the language of events, an \emph{event} is represented by some subset of the sample space.

\begin{example}
The event that both dice show a $1$ is the set $A = \{ (1,1) \}$.
\end{example}

\begin{example}
The event that the two dice show numbers which add to $3$ is the set $B = \{ (1,2), (2,1) \}$.
\end{example}

\begin{exercise}
Write down the event that at least one of the dice shows a $5$ as a set.
\end{exercise}

But how likely are these events?
For each event, we want to assign a number representing how common we think it is.
That is, for each subset of the sample space, we want to give a number representing the probability we get that event in a trial.

Mathematicians have settled on the following basic set-up: Probabilities should be numbers between $0$ and $1$, with $0$ representing that an event has basically no chance of happening, and $1$ meaning that the event will almost surely happen.

But how shall we make something reasonable? The \emph{frequentist interpretation} of probability is the idea that the probabilty of an event is the approximate proportion of times the event occurs when you run a very large number of trials.

\begin{quote}
If all outcomes of a game of chance are equally likely (i.e. the situation is unbiased), then the probability of an event $A$ is
\[ \mathrm{Pr}(A) =\dfrac{\text{number of elements in the event}}{\text{number of elements in the sample space}}\]
\end{quote}

\begin{example}
In our dice rolling example, the probability of the event that two $1$'s are up is
\[
\mathrm{Pr}(\text{two ones}) = \dfrac{1}{36}.
\]
This is because there is only one outcome in this event, and there are thirty-six total outcomes.
\end{example}

\begin{exercise}
Explain why the probability that the two numbers in our dice-rolling trial sum to 3 is $\frac{2}{36}$.
\end{exercise}

\begin{exercise}
Find the probability that at least one of the two dice shows up as a $5$.
\end{exercise}


\subsection*{Disjoint Events and the Addition Rule}

Two events are called \emph{disjoint} if the corresponding sets have no outcomes in common.
For example, the event that both dice show $1$ is disjoint from the event that the numbers on the dice sum to $3$.
In such a case, we have a simple rule for computing a special type of more complicated probability.
This is often called \emph{the addition rule}.

\begin{quote}
If two events $A$ and $B$ are disjoint, then the probability of the compound event $A$ or $B$ is the sum of the probability of $A$ and the probability of $B$.
\[
\mathrm{Pr}(\text{$A$ or $B$}) = \mathrm{Pr}(A) + \mathrm{Pr}(B)
\]
\end{quote}

Note that in mathematics we use the word ``or'' to mean ``this or that or both.''
This differs from spoken English, where sometimes ``or'' means ``this or that, but not both.'' 
Because we need to be clear about our language, mathematicians have settled on just this one version of the word ``or'' and we stick to it always.
Be careful!

Sometimes it is helpful to have notation for this, so we use the concept of the \emph{union} of two sets.
The union of $A$ and $B$ is the set which consists of all those things which are elements of $A$ or elements of $B$, or are elements of both sets.
\[
A \cup B = \{ x \mid \text{$x$ is an element of $A$ or $x$ is an element of $B$} \}
\]

\begin{example}
The probability that both dice show $1$ or that the dice sum to $3$ is 
\[
\mathrm{Pr}(A\cup B) = \mathrm{Pr}(A) + \mathrm{Pr}(B) = \dfrac{1}{36} + \dfrac{2}{36} = \dfrac{3}{36}
\]
\end{example}


\begin{exercise}
Write a few sentences to explain why the event that the numbers sum to $3$ is disjoint from the event that at least one $5$ is showing.

Then use the addition rule to compute the probabilty that the numbers sum to $3$ or at least one of the dice shows a $5$.
\end{exercise}


\subsection*{The Most Common Error}

Be careful with the addition rule!
The most common error is to forget to check that the events are disjoint, but to use the rule anyway.
If your events are not disjoint, then the addition rule will lead you in the wrong direction.

\begin{example}
Suppose that the two dice in our example are different colors so we can keep them apart clearly. Say, one die is red and the other blue.

Then the probability that the red die shows a $5$ is $1/6$.
The probability that the blue die shows a $5$ is $1/6$.
So you might be tempted to say that the probability that at least one of the two dice shows a $5$ is $1/6 + 1/6 = 1/3$.

This is incorrect. 
Why?
\end{example}


\subsection*{Complementary Events}

Two events are called \emph{complementary} when they are disjoint, but their union is the whole sample space.

For example, the event 
\begin{quote}
At least one die shows a $5$.
\end{quote}
and the event
\begin{quote}
Neither die shows a $5$.
\end{quote}
are complementary.
It is impossible to have an outcome which satisfies both events, but any outcome is in one of the two events.

The addition rule has a useful consequence:
\begin{quote}
If the events $A$ and $B$ are complementary, then
\[
\mathrm{Pr}(A)  = 1 - \mathrm{Pr}(B)
\]
\end{quote}

\begin{exercise}
Consider the event ``Neither of the dice shows a $5$.''
Compute the probability of this event in two ways:
\begin{compactitem}
\item Directly, using the definition of the probability.
\item Indirectly, using the rule about complementary events.
\end{compactitem}
You should get the same result from these two computations.
\end{exercise}



%\begin{thebibliography}{9}
%\end{thebibliography}

\end{document}
%sagemathcloud={"zoom_width":100}