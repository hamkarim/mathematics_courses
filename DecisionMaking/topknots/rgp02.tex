\documentclass[12pt,letterpaper]{article}

\usepackage[utf8]{inputenc}
\usepackage[T1]{fontenc}
\usepackage{amsmath}
\usepackage{amsfonts}
\usepackage{amssymb}
\usepackage{amsthm}
\usepackage[left=2cm,right=2cm,top=2cm,bottom=2cm,headheight=22pt]{geometry}
\usepackage{fancyhdr}
\usepackage{setspace}
\usepackage{lastpage}
\usepackage{graphicx}
\usepackage{caption}
\usepackage{subcaption}

\theoremstyle{definition}
\newtheorem{question}{Question}
\newtheorem{example}{Example}
\newtheorem{exercise}[question]{Exercise}
\newtheorem*{challenge}{Challenge}
\newtheorem{task}[question]{Task}

\begin{document}

%Paramètres de mise en forme des paragraphes selon les normes françaises
\setlength{\parskip}{1ex plus 0.5ex minus 0.2ex}
\setlength{\parindent}{0pt}

%Paramètres relatifs aux en-têtes et pieds de page.
\pagestyle{fancy}
\lhead{Theron J Hitchman}
\chead{\Large Reading and Guided Practice \#2}
\rhead{Fall 2013}
\lfoot{\emph{Math and Decision Making}}
\cfoot{}
\rfoot{\emph{\thepage\ of \pageref{LastPage}}}

\section*{Introduction}
In this assignment you will learn about how to see some physical properties of the picture hanging puzzle can be translated into properties of the symbols you learned about in previous work.
This should be helpful in trying to find a solution.

\section*{Goals}
At the end of this assignment, a student should be able to:
\begin{itemize}
\item state the effect of removing a nail from an attempted solution on the symbol associated to that attempt.
\item describe two situtations when attempted solutions are really the same physical solution, even though the symbols are different.
\end{itemize}
It is possible that a student might also:
\begin{itemize}
\item Solve the 2 nail picture hanging puzzle.
\end{itemize}

\section*{Reading and Questions for 30 August}

Now that we have encoded attempted solutions to the picture hanging puzzle into diagrams and then symbols, we want to use these symbols to help us solve the puzzle.
But first, we should note that there are several features of our symbol scheme we should understand more clearly.

\subsection*{Changing Direction}

The process for deciding on symbols depends upon a choice of ``direction of travel.''
For example, the diagrams for $AB$ and $B^*A^*$ are the same, except for the direction of travel along the curve.
\begin{task}
Draw the diagrams corresponding to the symbols $AB$ and $B^*A^*$. Verify that these correspond to the same attempt to solve the picture hanging puzzle, except for the  direction of travel.
\end{task}
Now, that one is pretty straightforward.
Let's check others.
\begin{exercise}
For each of the symbols below, \underline{draw} the corresponding diagram, and \underline{read the symbol} you get when you travel along the curve in the other direction.
\begin{itemize}
\item $ABAB$,
\item $AB^*A^*$,
\item $AB^*AA$.
\end{itemize}
\end{exercise}
So now you have several examples to think about.
(Psst. Do any of these solve the picture hanging puzzle? Check!)

\begin{question}
A single diagram has two symbols, one for each direction of travel along the string.
What is the relationship between the two possible symbols of a single diagram.
\end{question}

\subsection*{Removing a Nail}

An important part of the picture hanging puzzle is that you must remove nails.
What happens to the symbols involved?

There is no reason we cannot assign symbols to a configuration with only one nail. But if there is only one nail, we do not expect to see the label for the second nail in our symbol.
\begin{example}
Consider the diagram for the symbol $AB$.
(This is the ``standard'' way to hang a picture on two nails.)
If we remove the nail labelled $B$, then we get a configuration that should be labelled $A$.
(\textbf{Draw this to check it.})
\end{example}

\begin{exercise}
For each of the symbols below, \underline{draw} the diagram which corresponds, \underline{draw} the diagram with nail $A$ removed and \underline{find} the new symbol.
Then, \underline{draw} the diagram with the nail $B$ removed and \underline{find} the new symbol.
\end{exercise}

\begin{question}
What effect does removing a nail have on the symbols associated to a particular attempted solution to the picture hanging puzzle.
\end{question}

\subsection*{About those ${}^*$'s}

Something interesting happens when you remove nail $B$ from the configuration $ABA^*$.
The resulting configuration is $AA^*$.
\textbf{Draw this configuration.}
Do you see that $AA^*$ is a poor description?
If we pull the ends of the wire, that loop over nail A will shrink away!
The symbol $AA^*$ is the same as no symbol at all.
\begin{question}
What should the ``empty'' symbol represent, physically, in terms of the picture and how it hangs on the wall?
\end{question}

\begin{task}
Consider the three symbols below.
For each one, \underline{draw} the corresponding diagram.
Can the diagram be meaningfully simplified?
If so, write down the symbol for the simplified diagram.
\begin{itemize}
\item $ABB^*AB$
\item $BA^*ABAA^*$
\item $BB^*A^*A$
\end{itemize}
\end{task}

\begin{question}
Describe a rule for simplifying a symbol that encapsulates what you have just learned.
\end{question}

\subsection*{Challenge Question}

\begin{challenge}
Now that you have some more knowledge and experience, can you solve the 2 nail picture hanging puzzle? If you know one solution, can you find a second, genuinely different solution? How many solutions are there?
\end{challenge}

%\begin{thebibliography}{9}

 %\end{thebibliography}

\end{document}
%sagemathcloud={"zoom_width":100}